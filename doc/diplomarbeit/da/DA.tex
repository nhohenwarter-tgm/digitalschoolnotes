\documentclass[12pt]{article}

\usepackage[ngerman]{babel}
\usepackage[utf8x]{inputenc}
\usepackage{amsmath}
\usepackage{graphicx}
\usepackage[colorinlistoftodos]{todonotes}
\usepackage{listings}
\usepackage{glossaries}
\usepackage{placeins}
\usepackage{fixltx2e}
\usepackage{pdfpages}
\usepackage{lastpage}
\usepackage{enumitem}
\usepackage{xcolor}
\usepackage{scrpage2}
\usepackage{scrtime}
\usepackage{parskip}
\usepackage{hyperref}
\usepackage{titlesec}
\usepackage{multirow}

\clearscrheadfoot
\pagestyle{scrheadings}
\usepackage[
top    = 2.5cm,
bottom = 3cm,
left   = 3cm,
right  = 3cm]{geometry}
\setcounter{secnumdepth}{4}
\title{Diplomarbeit}


\author{DigitalSchoolNotes}
\date{\today}

%Paragraph Format
\titleformat{\paragraph}
{\normalfont\normalsize\bfseries}{\theparagraph}{1em}{}
\titlespacing*{\paragraph}
{0pt}{3.25ex plus 1ex minus .2ex}{1.5ex plus .2ex}

%Includes
\setlength{\parindent}{0cm}

\newcommand{\executeiffilenewer}[3]{%
\ifnum\pdfstrcmp{\pdffilemoddate{#1}}%
{\pdffilemoddate{#2}}>0%
{\immediate\write18{#3}}\fi%
}
\newcommand{\includesvg}[1]{%
\executeiffilenewer{#1.svg}{#1.pdf}%
{inkscape -z -D --file=#1.svg --export-pdf=#1.pdf --export-latex}%
\input{#1.pdf_tex}[width=1.0\textwidth]%
}

%Bibtex
\def\BibTeX{{\rm B\kern-.05em{\sc i\kern-.025em b}\kern-.08em
		T\kern-.1667em\lower.7ex\hbox{E}\kern-.125emX}}

%lslisting
\lstdefinelanguage{Javascript}{
  keywords={typeof, new, true, false, catch, function, return, null, catch, switch, var, if, in, while, do, else, case, break},
  keywordstyle=\color{blue}\bfseries,
  ndkeywords={class, export, boolean, throw, implements, import, this},
  ndkeywordstyle=\color{darkgray}\bfseries,
  identifierstyle=\color{black},
  sensitive=false,
  comment=[l]{//},
  morecomment=[s]{/*}{*/},
  commentstyle=\color{purple}\ttfamily,
  stringstyle=\color{red}\ttfamily,
  morestring=[b]',
  morestring=[b]"
}

\lstdefinestyle{customjava}{
  	language=Java,
  	frame=tlrb,
  	aboveskip=3mm,
  	belowskip=6mm,
  	showstringspaces=false,
  	columns=flexible,
  	basicstyle={\small\ttfamily},
  	numbers=none,
  	numberstyle=\tiny\color{gray},
  	keywordstyle=\color{purple},
  	commentstyle=\color{orange},
  	stringstyle=\color{blue},
  	breaklines=true,
  	breakatwhitespace=true
  	tabsize=3,
  	captionpos=b,
}



\lstset{escapechar=@,style=customjava}

%Cite right
\newcommand{\citeof}[2]{{
		\par \begingroup \leftskip=1cm \noindent \textit 
		''#1'' \cite{#2} \\
		\par \endgroup
	}}
	
% Picture insert (UseCase)
% \insertpicture{mik.png}{Some picture}{\cite{bk_key}}{itm:pic1}{0.5}
\newcommand{\insertpicture}[5]{{
	\begin{figure}[!htb]
		\centering\includegraphics[width=#5\textwidth]{#1}
		\ifthenelse{\equal{\unexpanded{#3}}{(selfmade)}}{\caption[#2]{#2}}{\caption[#2 #3]{#2 #3}}
		
		\label{#4}
	\end{figure}
	\FloatBarrier
}}
\makeglossaries
\newglossaryentry{json} {name=JSON, description={Java Script Object Notation}}
\newglossaryentry{Stakeholder} {name=Stakeholder, description={Eine Person welche Interesse am Ergebnis des Projektes hat, jedoch nicht an der Entwicklung des Projektes an sich beteiligt ist.}}
\newglossaryentry{Latenz} {name=Latenz, description={Verzögerung zwischen Anfrage und Antwort(Ping)}}
\newglossaryentry{TGM} {name=TGM, description={Technologisches Gewerbe Museum}}
\newglossaryentry{IDE} {name=IDE, description={Integrated Development Environment}}
\newglossaryentry{KVM} {name=KVM, description={Kernel-based Virtual Machine}}
\newglossaryentry{SSH} {name=SSH, description={Secure Shell}}
\newglossaryentry{DNS} {name=DNS, description={Domain Name Service}}
\newglossaryentry{HTTP} {name=HTTP, description={Hypertext Transfer Protocol}}
\newglossaryentry{HTTPS} {name=HTTPS, description={Hypertext Transfer Protocol Secure}}
\newglossaryentry{SSL} {name=SSL, description={Secure Sockets Layert}}
\newglossaryentry{API} {name=API, description={Application Programming Interface}}
\newglossaryentry{WSGI} {name=WSGI, description={Web Server Gateway Interface}}
\newglossaryentry{DSN} {name=DSN, description={Digital School Notes}}
\newglossaryentry{REST} {name=REST, description={Representational State Transfer}}
\newglossaryentry{CSS} {name=CSS, description={Cascading Style Sheets}}
\newglossaryentry{CAPTCHA} {name=CAPTCHA, description={Completely Automated Public Turing test to tell Computers and Humans Apart}}
\newglossaryentry{OAuth} {name=OAuth, description={Open Authentication}}
<<<<<<< HEAD
\newglossaryentry{Authentisierung} {name=Authentisierung, description={Authentisierung bedeutet Nachweis der behaupteten Identität.}}
=======
\newglossaryentry{PBKDF2} {name=PBKDF2, description={Password-Based Key Derivation Function 2}}
\newglossaryentry{CDN} {name=CDN, description={Content Delivery Network}}
\newglossaryentry{XPATH} {name=XPATH, description={XML Path Language}}
\newglossaryentry{SFTP} {name=SFTP, description={SSH File Transfer Protocol}}
>>>>>>> origin/development

\renewcommand*\glspostdescription{\dotfill}




\begin{document}


\begin{titlepage}
\begin{center}

% Deckblatt
\setlength{\arrayrulewidth}{0.3mm}
\begin{table}[]
\centering
\begin{tabular}{lcl}
\multirow{4}{*}{
\includegraphics[width=0.15\textwidth]{images/tgm_logo}} & \textbf{\Large Technologisches Gewerbemuseum}                  & \multirow{4}{*}{\includegraphics[width=0.12\textwidth]{images/htl_logo}} \\
                           & &\\ & \textbf{Höhere Lehranstalt für Informationstechnologie} &                   \\
                           & Ausbildungsschwerpunkt Systemtechnik                    &                  \\\hline
\end{tabular}
\end{table}

\textbf{\LARGE DIPLOMARBEIT}

\vspace{5mm}

Gesamtprojekt

\textbf{\Large DigitalSchoolNotes}

\vspace{15mm}

\end{center}

\textbf{User- und Rollenmanagement}\\
Philipp Adler \rule[-0.2cm]{1.3cm}{0pt} 5BHIT \hfill Betreuer: Michael Borko, Bakk. techn. \rule[-0.2cm]{0.4cm}{0pt}\\
\rule[-0.2cm]{9.8cm}{0pt} Mag. DI.(FH) Christoph Brein

\textbf{Datenmanagement}\\
Selina Brinnich \rule[-0.2cm]{1cm}{0pt} 5BHIT \hfill Betreuer: Michael Borko, Bakk. techn. \rule[-0.2cm]{0.4cm}{0pt}\\
\rule[-0.2cm]{9.8cm}{0pt} Mag. DI.(FH) Christoph Brein

\textbf{Infrastruktur und Testing}\\
Niklas Hohenwarter \rule[-0.2cm]{0.2cm}{0pt} 5BHIT \hfill Betreuer: Michael Borko, Bakk. techn. \rule[-0.2cm]{0.4cm}{0pt}\\
\rule[-0.2cm]{9.8cm}{0pt} Mag. DI.(FH) Christoph Brein

\textbf{Optical Character Recognition}\\
Adin Karic \rule[-0.2cm]{1.8cm}{0pt} 5BHIT \hfill Betreuer: Michael Borko, Bakk. techn. \rule[-0.2cm]{0.4cm}{0pt}\\
\rule[-0.2cm]{9.8cm}{0pt} Mag. DI.(FH) Christoph Brein

\textbf{Parallel Working System}\\
Thomas Stedronsky \rule[-0.2cm]{0.2cm}{0pt} 5BHIT \hfill Betreuer: Michael Borko, Bakk. techn. \rule[-0.2cm]{0.4cm}{0pt}\\
\rule[-0.2cm]{9.8cm}{0pt} Mag. DI.(FH) Christoph Brein

\vspace{25mm}

ausgeführt im Schuljahr 2015/16

\rule{1.0\textwidth}{0.3mm}\\
Abgabevermerk: 

Datum: \today \hfill \hfill übernommen von:\rule[-0.2cm]{3cm}{0pt}

\begin{center}

\noindent 

\vfill

\end{center}
\end{titlepage}


%HEADER AND FOOTER
\pagenumbering{Roman}
\ohead{\headmark}
\ifoot{© DigitalSchoolNotes - 2015/16}
\ofoot{\pagemark}

\newpage %----------------------------------------------------------------------------------------------

{\small\color{white}.}
\vspace{-0.7cm}
\renewcommand*\contentsname{Inhalt}
\tableofcontents

\newpage %----------------------------------------------------------------------------------------------
% Erklärung
%\includepdf[scale=1.2,pages=-]{documents/erklaerung}
\section*{Eidesstattliche Erklärung}
Ich erkläre an Eides statt, dass ich die vorliegende Diplomarbeit selbstständig und ohne fremde Hilfe verfasst, andere als die angegbenen Quellen nicht benutzt und die benutzten Quellen wörtlich und inhaltlich entnommenen Stellen als solche erkenntlich gemacht habe. \\

\rule[-0.2cm]{1cm}{0pt} Wien, am 31.03.2016 \\ \\

\hfill \hfill \rule[-0.2cm]{6cm}{0.5pt} \rule[-0.2cm]{2cm}{0pt}

\hfill \hfill Philipp Adler \rule[-0.2cm]{5.5cm}{0pt}\\ \\

\hfill \hfill \rule[-0.2cm]{6cm}{0.5pt} \rule[-0.2cm]{2cm}{0pt}

\hfill \hfill Selina Brinnich \rule[-0.2cm]{5.2cm}{0pt}\\ \\

\hfill \hfill \rule[-0.2cm]{6cm}{0.5pt} \rule[-0.2cm]{2cm}{0pt}

\hfill \hfill Niklas Hohenwarter \rule[-0.2cm]{4.4cm}{0pt}\\ \\

\hfill \hfill \rule[-0.2cm]{6cm}{0.5pt} \rule[-0.2cm]{2cm}{0pt}

\hfill \hfill Adin Karic \rule[-0.2cm]{6.0cm}{0pt}\\ \\

\hfill \hfill \rule[-0.2cm]{6cm}{0.5pt} \rule[-0.2cm]{2cm}{0pt}

\hfill \hfill Thomas Stedronsky \rule[-0.2cm]{4.4cm}{0pt}

\newpage

\vspace*{\fill} 
\section*{Gender Erklärung}

Aus Gründen der besseren Lesbarkeit wird in dieser Diplomarbeit  die Sprachform des generischen Maskulinums angewendet. Es wird an dieser Stelle darauf hingewiesen, dass die ausschließliche Verwendung der männlichen Form geschlechtsunabhängig verstanden werden soll.
\label{gender}
\vspace*{\fill}

\newpage

\section*{Abstract}

Nowadays most people can't think of a world without internet anymore. It is used almost always and everywhere you are - also in schools. Students are getting more and more into digital notes, especially in technical oriented schools, where they have their laptop always with them. There is only one problem: These digital notes are often written in different programs. They are unorganized and mostly left in some directory where they aren't opened once afterwards.\\
The project DigitalSchoolNotes was formed exactly because of this problem. Our webapplication can organize digital notes and make it easier to write them. It is made from students themselves to fit the needs of students optimal. The application can be accessed from almost every device - personal computers, laptops and even tablets. Furthermore it is possible to use a mobile phone to take a photo and put it into the digital notes easy and conveniently. Particularly fitted for technical schools there is also the possibility to insert some code-snippets into the digital notes, which are correctly formatted and highlighted.\\
All of which is to allow students to write tidy, organized, digital notes and to make learning from these easier and more efficient.
\section*{Kurzfassung}
\cfoot{Selina Brinnich}

%Text...


\newpage

\section*{Danksagung}
\cfoot{}
Wir wollen uns zunächst bei jenen Menschen bedanken, die auf ihre herzliche und unterstützende Art an der erfolgreichen Umsetzung dieses Diplomprojekts beigetragen haben. Ohne sie hätten wir bei manchen Problemen vielleicht doppelt so lang gebraucht und hätten wahrscheinlich nur halb so viel Spaß an der Umsetzung gehabt.

Wir wollen vor allem auch unseren Familien und Freunden danken. Im Zuge dieses Diplomprojektes sind einige Herausforderungen auf uns zugekommen, die wir ohne die Motivation und Unterstützung aus dem Familien- und Freundeskreis nicht so gut gemeistert hätten. Wir möchten besonders unseren Eltern danken, die uns auch nach langen Arbeitssitzungen stets wieder aufgerappelt haben.

Ein besonderer Dank geht an unsere beiden Betreuer, Prof. Borko und Prof. Brein. Ohne die zielgerichteten Verbesserungsvorschläge und die technische Beratung von Prof. Borko wäre unser Weg zur Fertigstellung des Produkts ein deutlich längerer gewesen. Prof. Brein stand uns mit seiner langjährigen Erfahrung und benutzerorientierten Sicht zur Seite.

Abschließend wollen wir uns auch noch bei unserem äußerst motivierten Abteilungsvorstand Prof. Koppensteiner und der Abteilung für Informationstechnologie bedanken. Der Abteilungsvorstand hatte immer ein offenes Ohr für uns, egal mit welchen Anliegen wir zu ihm kamen. Wir haben uns, im Zuge unserer fünfjährigen Ausbildung am TGM, immer als ein Teil dieser Familie gesehen und werden auch in Zukunft mit Stolz auf unsere Zeit hier zurückblicken.

\label{pageRomanEnd}

\newpage %----------------------------------------------------------------------------------------------

\pagenumbering{arabic}
\ofoot{\pagemark}

\section{Einleitung}
\label{sec:einleitung}
%\section*{Einleitung}
\cfoot{Niklas Hohenwarter}

%Text...

\section{Problembeschreibung}
\label{sec:problembeschreibung}
%\section*{Problembeschreibung}
\cfoot{Niklas Hohenwarter}

Im Schulunterricht mitzuschreiben ist wichtig, jedoch immer weniger Schüler machen dies auch. Falls sie doch mitschreiben, dann verschwindet der Zettel oder Collegeblock meistens oder wird nie wieder angesehen. Es ist kompliziert den in immer größeren Teilen digital ablaufenden Unterricht auf Papier festzuhalten und seine Mitschriften zu organisieren. Als Resultat dieser Umstände stirbt die händisch geführte Mitschrift langsam aber sicher aus. \\

Statt der händischen Mitschrift wird die digitale Mitschrift am im Unterricht verwendeten Laptop immer beliebter. Diese hat allerdings bis heute ähnliche Probleme. Es ist kompliziert die Struktur des Tafelbildes in z.B. ein Word-File zu übernehmen. Außerdem passiert es häufig, dass die Datei welche die Mitschrift enthält nicht mehr gefunden wird oder einfach einen unpassenden Dateinamen hat, welcher nicht die entsprechende Mitschrift vermuten lässt. Mit aktuellen Produkten ist es ebenfalls mühsam digital mitzuschreiben.

\subsection{Umfeldanalyse}
\label{subsec:umfeldanalyse}
%\section*{Umfeldanalyse}
\cfoot{Niklas Hohenwarter}

Es existiert bereits einiges an Software, welche sich mehr oder weniger dafür eignet digital mitzuschreiben. Die Features der einzelnen Produkte sind großteils bekannt weshalb diese nur oberflächlich beschrieben werden:

\begin{itemize}
\item \textbf{Microsoft Word:} Am weitesten verbreitete Textverarbeitungssoftware; kostenpflichtig; auf Windows, Mac, Android \& iOS verfügbar
\item \textbf{Libre Office Writer:} Textverarbeitung; OpenSource; kostenlos; auf Windows, Mac \& Linux verfügbar
\item \textbf{OpenOffice Writer:} Textverarbeitung; OpenSource; kostenlos; auf Windows, Mac \& Linux verfügbar
\item \textbf{Kingsoft WPS Writer:} Textverarbeitung; kostenlos oder kostenpflichtig; auf Windows, Mac, Android \& iOS verfügbar; sehr stark an Microsoft Word angelehnt
\item \textbf{Microsoft OneNote:} Notizprogramm; Notizen können beliebig platziert und angeordnet werden; Organisation in Notizbüchern; gleichzeitiges Bearbeiten möglich; auf Windows, Mac, Android \& iOS verfügbar; kostenlos
\item \textbf{Google Docs:} Textverarbeitung; kostenlos; Betriebssystemunabhängig (Browser); gleichzeitiges Arbeiten möglich
\end{itemize}

\newpage

\subsection{Projektidee}
\label{subsec:projektidee}
%\section*{Projektidee}
\cfoot{Adin Karic}

%Text...

\subsection{Projektkoordination}
\label{subsec:projektkoordination}
%\section*{Projektkoordination}
\cfoot{Adin Karic}

%Text...
\subsubsection{Kurzeinführung in Scrum}

\subsubsection{Scrum im Team}

\newpage
%----------------------------------------------------------------------------------------------

\section{Stand der Technik}
\label{sec:standdertechnik}
%\section*{Stand der Technik}
\cfoot{Philipp Adler}

Die gloable, schnelllebige IT Welt bietet verschiedenste Technologien, die Entwickler in ihrer Aufgabenstellung unterstützen. Bei diesem Überangebot ist es schwierig, das Passende für seine Anforderung, zu finden.\\
Dieses Kapitel befasst sich mit dem Vergleich nützlicher Systeme, um sich für die Geeignetsten zu entscheiden.

\subsection{Frameworks}
\label{subsec:frameworks}
%\section*{Frameworks}
\cfoot{Philipp Adler}

Frameworks sind eine Sammlung von Softwareprodukten, die Entwicklungsansätze zur Verfügung stellen. Sozuagen ermöglichen Sie durch vordefinierte Grundbausteine eine schnelle, reibungslose Entwicklung. Mithilfe von Schnittstellen und Bibliotheken soll dem Entwickler Arbeit abgenommen werden.\\
Im Falle von DSN fungieren die Frameworks als Rundumpaket. Für unser System benötigen wir:
\begin{itemize}
\item \textbf{Web Framework} als Grundgerüst
\item \textbf{JS-Framework} um auf Useraktionen zu reagieren
\item \textbf{CSS-Framework} für das grafische Design
\item \textbf{Element-Framework} als Funktionaltität in den Schulheften
\item \textbf{GUI-Testing} um zu garantieren, dass alle Funktionen reibungslos laufen
\end{itemize}

\subsubsection{Web Frameworks}
Web Frameworks haben das Ziel, dem Entwickler einer Webanwendung, mit Hilfe von vordefinierten Klassen, zu unterstützen. Sie sorgen dafür, dass eine Verbindung mit der Datenbank aufgebaut wird, Inhalte dynamisch angezeigt werden und ein Anmeldesystem zur Verfügung steht.\\
Web Frameworks teilen das Projekt in 2 Teile, dem Backend und Frontend.\\
Für die Auswahl unserer Web Frameworks wurden Flask, Django und Play verglichen. Dabei wurde darauf geachtet, wie groß der Aufwand der Installation und Konfiguration ist, wie weit es ausbaufähig ist und welche Funktionalität bereits zur Verfügung steht.

\newpage

\paragraph{Flask}
\grqq{}Flask ist ein microframework für Python, basierend auf Werkzeug, Jinja 2.\grqq{}\cite{FLASK} Es ist frei verfügbar und aufgrund seiner Größe, schnell und einfach zu installieren und konfigurieren. Da es zu den kleineren Web Frameworks zählt, wird wenig Funktionalität geboten, welche aber durch einfache Erweiterungen behoben werden kann.\cite{FLASK}

Die Installation von Flask ist ziemlich simpel. Es kann mit folgenden Befehl installiert werden:
\begin{lstlisting}[caption={Installation von Flask \cite{FLASK}}]
pip3 install Flask
\end{lstlisting}

Für ein einfaches \grqq{}Hello World\grqq{}-Programm, geschrieben in Flask, braucht es eine Funktion, die die Message ausgibt und eine Main-Methode, welche die Applikation startet.

\begin{lstlisting}[caption={Flask Hello-World \cite{FLASK}}, escapeinside={(*}{*)}]
from flask import Flask

app = Flask(__name__)

@app.route("/")
def hello():
	return "Hello Flask World!"

if __name__ == "__main__":
    app.run()
\end{lstlisting}

Mit dem Befehl \textit{python3 hello.py} lässt sich das Programm ausführen.

\paragraph{Django}
Django ist ein Open-Source Python Web Framework für die schnelle Entwicklung einer Webanwendung. Es bietet eine detailierte und umfangreiche Dokumentation über alle Funktionalitäten und wird außerdem durch große Community unterstützt. Obwohl es mit sehr vielen Features ausgestattet ist, ist es einfach erweiterbar und kann mit wenig Aufwand installiert und konfiguriert werden.\cite{DJANGO}

Installiert wurde das Web Framework folgenderweise:
\begin{lstlisting}[caption={Installation von Django\cite{DJANGOIN}}]
apt-get install python3-django
\end{lstlisting}

Zur Erstellung eines \grqq{}Hello-World\grqq{}-Programmes muss zu Beginn ein Projekt erzeugt werden. Im nächsten Schritt ist die Datenbank entsprechend der Project-Settings zu konfigurieren. Anschließend wird im Unterordner app ein neues File \grqq{}hello.py\grqq{} mit folgenden Code erstellt:

\begin{lstlisting}[caption={Django Hello-World \cite{DJANGOCODE}}]
from django.http import HttpResponse

def hello_world(request):
	return HttpResponse("Hello Django World!")
\end{lstlisting}

Wird der Server mit dem Befehl \textit{python manage.py runserver} gestartet, erscheint die Message im Browserfenster.

\paragraph{Play}
Play, ein leichtgewichtiges Web Framework, basierend auf Java und Scala, steht für den kommerziellen Gebrauch kostenlos zur Verfügung. Allerdings ist die offizielle Dokumentation nicht besonders ausführlich.\\
Trotz vorhandener Ausbaufähigkeit, erwies sich die Umsetzung als kompliziert. Die Installation gestalltete sich zeitaufwendiger und dauerte im Vergleich zu anderen Web Frameworks länger.\cite{PLAY}

Um Play zu installieren muss das Framework zunächst von\\
https://www.playframework.com/download heruntergeladen werden. Das downgeloadete zip-File muss entpackt und der darin enthaltene Activiator zum PATH hinzugefügt werden. Danach wird der Activiator mit folgendem Befehl gestartet\cite{PLAYCON}:
\begin{lstlisting}[caption={Konifiguration von Play \cite{PLAYCON}}]
activiator ui
\end{lstlisting}

Für das \grqq{}Hello-World\grqq{}-Programm braucht es ein neues Projekt. Im Unterordner /controllers muss folgende Methode hinzugefügt werden:

\begin{lstlisting}[caption={Play Hello-World \cite{PLAYCON}}]
public static Result hello() {
	return ok(main.render("Hello World",new
		play.twirl.api.Html("Hello Play World!")));
}
\end{lstlisting}

Zu guter Letzt wird der Prototyp mit dem Befehl \textit{activator run} im Projekt-Verzeichnis ausgeführt.

\paragraph{Vergleich}
Das DSN-Team hatte im Vorfeld mit der Programmierung von Web Frameworks noch keine Erfahrungswerte. \\
Aufgrund der leicht verständlichen Dokumentation und der gut vernetzten Community, hat sich Django klar herauskristallisiert. Als wichtige Entscheidungshilfe galten die erstellten Prototypen. Die schnelle, einfache Entwicklung, sowie die Installation und Konfiguration des Frameworks, hinterließ beim Team einen guten Eindruck.

\subsubsection{JS-Frameworks}
JavaScript-Frameworks sind dazu da, um Userinteraktionen entgegenzunehmen, diese Daten zu validieren, verarbeiten und am Ende das Ergebnis zu retournieren.\\
Im Falle von DSN, dienen Sie als Schnittstelle zum Webserver und außerdem, dazu, den Content dynamisch zu ändern. Verglichen wurden die bekanntesten, populärsten JavaScript-Frameworks in den Bereichen:
\begin{itemize}
\item Browserunterstützung
\item Dokumentation \& Community
\item Lizenz und Kosten
\item Schwierigkeit bei der Erstellung eines Prototypen
\end{itemize}
\paragraph{Dojo}
Dojo ist eine JS-Bibliothek von Dojo Foundation, die Entwicklern JS- und Ajax-basierende Module anbietet. Das Framework ist modular aufgebaut, das für eine gute Übersicht im Code sorgt. Dojo ist frei anwendbar und plattformunabhängig. Die Dokumentation besteht aus einer übersichtlichen Auflistung von step-by-step Tutorials. \cite{DOJO}

Das Installieren bzw. Anwenden des Frameworks ist sehr einfach. Eine Möglichkeit besteht darin, den gesamten Source-Ordner herunterzuladen, in dem sich der Core mit den Kernfunktionalitäten befindet oder ganz einfach den Link der dojo.js Datei im Header angeben.
\begin{lstlisting}[caption={Dojo einbinden\cite{DOJODOWN}}]
<script src="//ajax.googleapis.com/ajax/libs/dojo/1.10.4/dojo/dojo.js">
</script>
\end{lstlisting}

Der Prototyp, ein Drag \& Drop Beispiel, ließ sich dank verständlicher Erklärung, ohne Probleme umsetzen. Allerding muss sich der Entwickler im Klaren sein, welche Module für eine lauffähige Applikation einzubinden sind. \cite{DOJOINFO}

\paragraph{Sencha ExtJS}
Das ExtJS JavaScript Framework von Sencha, dient der Realisierung von komplexeren Anwendungen. Sencha bietet eine gut ausgebaute, strukturierte API. Neben einer großen Community von über 500.00 Mitgliedern, unterstützt ExtJS alle marktführenden Webbrowser, sowie Smartphones.\\
Sencha ist unter der Commercial License oder GNU General Public License verfügbar. Für den kommerziellen Nutzen benötigt es eine Lizenz im Wert von \$895.00.\cite{SENCHA}

Sencha ExtJS hat den Vorteil, dass es unabhängig ist, ohne Backend ausgeliefert wird und viele fertige Komponenten (Charts, Grids, Forms) schon vorhanden sind. Basierend auf einem MVC-Modell ist das Klassensystem objektorientiert aufgebaut. Für das Testing bietet die Sencha Suite keine eigenen Frameworks, aber es besteht die Möglichkeit, mit verschiedenen Drittanbieter wie \grqq{}Siesta, Jasmine, Mocha\grqq{} zu testen.\cite{SENCHAFEATURES}\cite{SENCHALICENSE}

Für die Verwendung von Sencha müssen die benötigiten Module bzw. Library im Header des HTML-Files eingebunden werden. Die Umsetzung des Prototyps gestaltete sich einfach und konnte den Bedürfnissen entsprechend angepasst werden. Da Sencha HTML und JavaScript skrikt trennt, ist nur die Einbindung von fertigen JS-Files notwendig.

\paragraph{jQuery}
jQuery, ist ein plattformunabhängiges JavaScript Framework, von der jQuery Foundation. Die Open-Source-Software steht frei für jegliche Verwendung zur Verfügung, solange im Projekt überall die Copyrights stehen. Mit einer umfangreiche Klassenbibliothek unterstützt jQuery den Umgang mit DOM (Document Object Model). DOM wandelt alle HTML-Elemente in Objekte um, die während der Laufzeit dynamisch angepasst werden.\cite{JQUERY} 

Für die Nutzung ist die jQuery.js Datei in den Header einzubinden.
\begin{lstlisting}[caption={jQuery einbinden\cite{JQUERYDOWN}}]
<script src="//code.jquery.com/jquery-1.12.0.min.js"></script>
\end{lstlisting}
Nebenbei sei erwähnt, dass mittels dem '\$' Zeichen die Funktionen aufgerufen werden.

Für den Entwickler ist jQuery ein leicht verständliches Framework, mit einer Vielzahl von Funktionen. In kurzer Zeit ist es möglich große Fortschritte zu erzielen. So war auch die Erfahrung bei der Umsetzung des Prototyps. Hauptsächlich musste nur eine Funktion geschrieben werden, welche auf eine id zeigt. \cite{JQUERYTOOL}

\paragraph{AngularJS}
AngularJS ist ein JavaScript Framework von Google, welches einen MVC-Ansatz verfolgt. Die 4 wichtigisten Browser: Chrome, Firefox, Safari und IE werden unterstützt. Obwohl die Community relativ wenig Mitglieder beherbergt, beinhaltet die Dokumentation ausführliche Informationen über Funktionen, die zusätzlich mit Beispielen untermauert sind. Als OpenSource Framework, unter der MIT Lizenz veröffentlicht, erlaubt es die freie Nutzung und Veränderung des Codes.

Für die Verwendung des Frameworks wird der HTML-Code einfach um AngularJS-Attribute erweitert. Dadurch besteht eine strikte HTML und JS Trennung, welches die Lesbarkeit, Übersichtlichkeit und Testbarkeit des Cods verbessert. \cite{ANGULARJS}

Um AngularJS in der Praxis einsetzen zu können, muss die JS-Bibliothek im Header inkludiert werden: 
\begin{lstlisting}[caption={AngularJS einbinden\cite{ANGULARJSDOWN}}]
<script 
src="https://ajax.googleapis.com/ajax/libs/angularjs/1.5.2/angular.min.js">
</script>
\end{lstlisting}

Verpflichtend für die Anwendung von AngularJS ist, dass setzen des ng-app Attribut im HTML-Root Tag. Es soll dem Framework mitteilen, wo sich das Root-Element befindet. AngularJS wird im HTML Code in geschwungenen Klammern definiert.\\
Anfangs gestaltete sich die Implementation der Drag \& Drop Applikation als kompliziert. Doch nach zunehmender Erfahrung mit dem Framework, konnte in sehr kurzer Zeit viel erreicht werden.

\paragraph{Meteor}
Meteor ist ein full-stack JavaScript Framework, welches alle Plattformen unterstützt. Erwähnenswert ist, dass Meteor mit NoSQL Datenbanken, insbesondere MongoDB kooperiert. Ähnlich wie Java, besitzt es eine große, strukturierte API, dessen einzelne Methoden ausführlich beschrieben sind. Die Nutzung von Meteor ist frei. Dank einer eigenen Community, namens Meteorpedia, steht einer Problemlösung nichts im Weg. Ein weiterer Vorteil ist das live deploying. Dadurch wird dem Entwickler möglich, Änderungen automatisch in Jetztzeit hochzuladen und zu deployen. \cite{METEOR}

\newpage

Die Installation dauerte im Gegensatz zu anderen JavaScript Frameworks länger. Es gab nicht die Möglichkeit einer Header-Einbindung, sondern musste mit folgenden Befehl installiert und getestet werden:
\begin{lstlisting}[caption={Installation von Meteor \cite{METEORINSTALL}}]
curl https://install.meteor.com/ | sh
meteor create ~/my_cool_app
cd ~/my_cool_app
meteor
\end{lstlisting}

\textit{meteor create} erstellt ein neues Meteor Projekt. Durch die Angabe von $\sim$/my\_cool\_app wird ein default Projekt erstellt, das gleichzeitig als Prototyp genutzt werden kann. Bei diesen Prototypen handelt es sich um einen Klickzähler, der bei jedem Klick die Variable um 1 erhöhte und anschließend ausgab.

\paragraph{Vergleich}
Wichtig für das DSN Team ist es, die Kosten so gering wie möglich zu halten, sowie das Vorhandensein einer ausführliche Dokumentation, zum besseren Verständnis mit Beispielen untermauert. Die Entscheidung fiel zugunsten AngularJS und jQuery, weil es neben den oben genannten Forderungen viele Funktionalitäten anbietet und mit dem Web Framework Django harmoniert.

\subsubsection{CSS-Frameworks}
Nicht nur das Backend spielt im System eine wichtige Rolle, sondern auch Design und Gestaltung einer Website. CSS-Framework 
spricht User an, denen Klarheit und Übersichtlichkeit wichtig ist. Sozusagen interagiert es mit dem Anwender und animiert ihn, das System zu verwenden. Ein Framework soll den Entwicklern Arbeit abnehmen und dabei helfen, sich auf die wichtigen Dinge zu konzentrieren.\\
Wichtig erscheint uns, dass ein CSS-Framework:
\begin{itemize}
\item einfach zu installieren und konfigurieren ist
\item browserunabhängig und sich dynamisch an Devices anpasst
\item zahlreiche, vordefinierte Funktionen bietet, die ausführlich dokumentiert sind
\end{itemize}

\newpage

\paragraph{Yaml}
Yaml ist ein CSS-Framework von Dirk Jesse, welches von allen modernen Browsern, wie Chrome, Firefox, Opera, Safari und Internet Explorer, unterstützt wird. Es passt sich an jeden Screen dynamisch an, sei es bei diversen Browsern, als auch auf mobilen Endgeräten wie iPhone und iPad. Hinzuzufügen ist, dass das Framework modular aufgebaut ist. Neben dem Kernmodul, welches flexible Layouts, variable Spaltenbreiten, sowie Grid-Layouts mit fester Breite beinhaltet, können nach Belieben weitere Module eingebunden werden. Yaml bietet eine umfangreiche deutschsprachige Dokumentation, die alle notwendigen Informationen enthält. Leider ist die Suche nach bestimmten Elementen nicht möglich. \cite{YAML}

Das Framework wird unter der Creative Commons Attribution 2.0 Lizenz (CC-BY 2.0) veröffentlicht, die privaten als auch kommerziellen Gebrauch erlaubt.

Die aktuellste Version kann von der Hauptseite http://www.yaml.de heruntergeladen und entpackt werden. Im entpackten Ordner befinden sich alle notwendigen .css Dateien und Demos, die für den Prototyp essentiell sind. Bei der Verwendung muss in den HTML Files der Pfad zum CSS File und bei HTML Tags die gewünschte class angegeben werden. 
\begin{lstlisting}[caption={YAML einbinden \cite{YAMLPROTO}}]
<link rel="stylesheet" href="yaml/core/base.css" type="text/css"/>
<link rel="stylesheet" href="css/styles.css" type="text/css"/>
\end{lstlisting}

Dadurch wird automatisch nach dem Style des CSS-Frameworks formatiert.\\
So ließ sich in sehr kurzer Zeit ein userfreundliches Anmeldeformular erstellen.

\paragraph{Pure}
Pure ist eine nur 4kb große, kombrimierte CSS-Datei, dass heißt im Vergleich zu anderen sehr klein. Hauptsächlich besteht es aus einem Set von CSS Modulen. Die Dokumentation ist übersichtlich gegliedert, sodass sich jeder schnell zurechtfindet. Mithilfe des Navigationsbalken können bestimmte Elemente schnell gefunden werden. Bei Auftreten von Problemen mit Pure stehen zahlreiche Foren zur Verfügung. Das Framework wird unter der BSD-Lizenz veröffentlicht, und kann kostenlos genutzt werden.

Der Unterschied zu anderen CSS-Frameworks wie Bootstrap oder Yaml ist, dass es ohne JavaScript Plugins ausgeliefert wird. JavaScript wird nur für Dropdown-Menüs und fixed top- bzw. bottom-Navigation eingesetzt. Pure bietet sechs Bausteine, welche die Hauptanforderungen eines jeden Entwicklers abdecken. Da Pure wegen der kleinen Größe nur die notwendige Funktionalität anbietet, ist es bei Bedarf erweiterbar.

Für die Verwendung von Pure muss hauptsächlich der Link, zu dem css. File im Header des HTML-Files eingefügt werden. \cite{PURE}
\begin{lstlisting}[caption={Pure einbinden \cite{PURE}}]
<link rel="stylesheet" href="http://yui.yahooapis.com/pure/0.6.0/pure-min.css">
\end{lstlisting}

Der Prototyp, ein Anmeldeformular, konnte durch das gut beschriebene Tutorial, untermauert mit Beispielen, sehr schnell und leicht nachgebildet werden.

\paragraph{Bootstrap}
Das frei verfügbare CSS-Framework Bootstrap von Yahoo unterstützt alle Elemente, die essential für eine Webseite sind. Dem Entwickler steht es frei, welche Komponenten er verwenden möchte. Bootstrap passt sich dynamisch an jedes Device an. Es ist plattformunabhängig und bietet eine sehr genaue Dokumentation mit Navigationsbalken.

Für den Einsatz, muss Bootstrap von der offiziellen Seite http://getbootstrap.com heruntergeladen werden. Es ist nicht verpflichtend, die komplette Bibliothek zu übernehmen, sondern Komponenten, je nach Bedarf.
 
Die Installation bezieht sich lediglich auf das Einbinden der bereitgestellten Dateien in das eigene Projekt. Bootstrap wird als ein ZIP-Archiv bereitgestellt. In diesem befindet sich eine CSS-Datei und eine Javascript Datei. Die beiden Dateien müssen anschließend in den $<$head$>$ der Webseite eingebunden werden. \cite{BOOTSTRAP}
\begin{lstlisting}[caption={Bootstrap einbinden \cite{BOOTSTRAP}}]
<link rel="stylesheet"
href="https://maxcdn.bootstrapcdn.com/bootstrap/3.3.5/css/bootstrap.min.css">
<script src="https://maxcdn.bootstrapcdn.com/bootstrap/3.3.5/js/bootstrap.min.js">
</script>
\end{lstlisting}

\paragraph{Vergleich}
Aufgrund der erstellten Prototypen und der ausführlichen Dokumentation, fiel die Entscheidung auf Bootstrap. Bootstrap unterstützt alle Browser, passt sich dynamisch dem Bildschirm an und ist mit wenig Aufwand einsetzbar. Die Syntax des CSS-Framework ist klar verständlich und hat außerdem den Vorteil, dass es viele Möglichkeiten der Formatierung und Validierung gibt.

\subsubsection{Element-Frameworks}
In diesem Kapitel werden alle notwendigen Elemente dargestellt, die in den Schulheften der DSN-Usern zur Verfügung stehen. Gemeint sind das Codeelement, Textelement, Zeichentool und die Bildergalerie.\\
Dem DSN-Team ist es wichtig, dass sich die Elemente den Anforderungen entsprechend anpassen können.

\paragraph{CodeMirror}
CodeMirror ist ein Texteditor basierend auf JavaScript, welcher dank MIT Lizenz kommerziell genutzt werden darf. CodeMirror wird von Firefox, Chrome, Safari, IE und Opera unterstützt. Durch die Einbindung von Frameworks ist es möglich, ein Textfeld darzustellen. Dank unterschiedlicher Sprachunterstützungen, erkennt das System automatisch Syntax und hebt diese farbig hervor. Zur besseren Übersicht, werden am linken Rand Zeilennummern eingeblendet. \cite{CODEMIRROR}

\insertpicture{images/framework/CodeMirror}{CodeMirror}{\cite{CODEMIRROR}}{itm:codemirror-chart}{0.55}

\paragraph{Sketch.js}
Sketch.js ist ein freiverfügbares Zeichentool. Diverse Parameter erlauben dem User, Farbe und Pinselgröße auszuwählen. Außerdem ,ermöglicht das plattformunabhängige Framework, einen reibungslosen Download der angefertigten Zeichnung im PNG- oder JPEG-Format. \cite{SKETCH}

\insertpicture{images/framework/Sketch}{Sketch.js}{\cite{SKETCH}}{itm:sketch-chart}{0.55}

\paragraph{CKEditor}
Der CKEditor von CKSource ist ein Texteditor, der sich ohne Mühe in Websiten einbinden und verwenden lässt. Als vorteilhaft erweist sich die Kompatiblität mit den bedeutesten Webbrowsern. Durch dieses Framework ist es möglich, einen Textblock zu erstellen und diesen individuell zu bearbeiten.

Außerdem ist es möglich, seinen CKEditor an die eigenen Bedürfnisse anzupassen.\cite{CKEDITOR}

\insertpicture{images/framework/CKEditor}{CKEditor}{\cite{CKEDITOR}}{itm:ckeditor-chart}{0.55}

\paragraph{Photoswipe}
Das Framework Photoswipe kann Bilder in einer Slideshow anzeigen und durch diese navigieren. Photoswipe ist auf allen Browsern vertreten, hat gute Grundstrukturen, braucht aber für unsere Zwecke ein paar kleine Anpassungen. \cite{PHOTOSWIPE}

\insertpicture{images/framework/Photoswipe}{Slideshow}{\cite{PHOTOSWIPE}}{itm:photoswipe-chart}{0.55}

\newpage

\subsubsection{GUI-Testing}
Um eine Software auf seine einwandfreihe Funktion zu testen, wird ein geeignetes GUI-Testing Framework benötigt. Die Aufgabe besteht darin automatisiert Fehler ausfindig zu machen, mögliche Benutzerinteraktionen zu testen und auch grafische Elemente zu überprüfen. 

\paragraph{Sahi OS}
Sahi OS ist ein automatisiertes Open Source Testing Framework. Neben der kostenlosen Version mit einer eingeschränkte Anzahl an Features, gibt es eine Kostenpflichtige, mit umfangreichen Features. Die Testfälle lassen sich nur auf Firefox, Chrome und IE ausführen. Bei auftreten von Fehlern, hilft die gut beschriebene, aber etwas unübersichtliche Dokumentation.

Die Installation erwies als sehr einfach. Auf http://sahi.sourceforge.net/install.html\#install gibt es einen Installer zum Download, welcher das Tool auf Port 9999 startet. Obwohl kaum Probleme beim Erstellen der Testfälle mittels GUI Tool und per Script aufgetreten sind, ließ sich der Prototyp nicht ausführen. Die sehr kleine Community konnte uns auch nicht weiterhelfen.

\paragraph{Watir}
Watir ist ein Testing-Tool basierend auf Ruby. Egal, in welcher Sprache das Projekt geschrieben oder welcher Browser verwendet wird, Watir unterstützt es.\\
Die Dokumentation war schwerig zu finden und ist außerdem sehr unübersichtlich. \cite{WATIR}

Um Watir zu installieren müssen folgende Befehle in der Konsole ausgeführt werden: 
\begin{lstlisting}[caption={Installation von Watir \cite{WATIRINSTALL}}]
sudo apt-get install ruby ruby-dev
sudo apt-get install rubygems
gem update --system --no-rdoc --no-ri
gem install watir --no-rdoc --no-ri
\end{lstlisting}


Die Ausführung der Tests war aus 2 Gründen nicht umsetzbar. Zum einen ließ sich Rubygem Watir-Web-Framework nicht installieren. Zum anderen konnten die Testfälle aus unerklärlichen Gründen nicht ausgeführt werden.

\paragraph{Robot Framework}
Das Testing Framework, Robot Framework ist auf Akzeptanztests ausgelegt. Es ist frei unter Apache 2.0 Lizenz verfügbar. Die Dokumentation ist viel zu unübersichtlich und die Beispiele werden nicht step-by-step erklärt, was die Umsetzung des Prototypen unmöglich machte. \cite{ROBOTFRAMEWORK}

Die Installation verlief unter Python mit den Befehlen:
\begin{lstlisting}[caption={Installation von Robot Framework \cite{ROBOTFRAMEWORKINSTALL}}]
sudo apt-get install python-pip
sudo pip install robotframework
sudo pip install docutils
\end{lstlisting}

Obwohl die Installationsanleitung korrekt befolgt wurde, konnten bestimmte Librarys nicht gefunden werden. Fazit: die Tests konnten nicht ausgeführt werden. 
\paragraph{Selenium}
Eines der bekanntesten freien Testing Frameworks ist Selenium. Selenium unterstützt die meisten Browser und besteht aus einer großen Community. Die Testfälle können auf allen gängigen Programmiersprachen erzeugt werden.

Die Bibliothek lässt sich entweder als stand-alone Software oder als Plugin in einer IDE z.B. Eclipse installieren. Für die Ausführung müssen lediglich die Bibliotheken in das Projekt eingebettet werden. Es besteht die Möglichkeit die Tests mittels GUI Tool oder per Script zu erstellen.\\
Innerhalb von 5 Minuten funktionierte alles reibungslos.

\paragraph{Vergleich}
Da Selenium als einziges Produkt innerhalb kürzester Zeit den gewünschten Erfolg brachte, entschied sich das Team für dieses. Desweiteren bietet es eine sehr große Community und lässt sich komplett automatisieren und in CI Tools integrieren.

\newpage

\subsection{Technologien}
\label{subsec:technologien}
%\section*{Technologien}
\cfoot{Selina Brinnich}

%Text...

\newpage %----------------------------------------------------------------------------------------------

\section{Design}
\label{sec:design}
%\section*{Design}
\cfoot{Philipp Adler}

%Text...

\subsection{Software-Architektur}
\label{subsec:softwarearchitektur}
%\section*{Software-Architektur}
\cfoot{Philipp Adler}

Die Software-Architektur soll die Bauweise eines Systems abbilden und stellt den Ausgangspunkt eines erfolgreichen Systems dar. Es beschreibt die Vernetzung der Software- und Hardwaresegmente. Des Weiteren spielt die Platzierung, sowie die Zusammenarbeit und Anordnung der Softwarekomponenten eine wichtige Rolle.\\
Welche Schnittstellen und Beziehungen stehen zwischen den Elementen, wie findet die Interaktionen zwischen Client und Server statt? All das sind Fragen, die bei der Entwicklung eines Systems bedacht werden müssen. \cite{VERTEILTE_SYSTEME}

\subsubsection{Ablauf}
DSN ist eine Web-Applikation, welche den Benutzern, im Speziellen Schülern, helfen soll, seine Mitschriften organisierter und einfacher zu verwalten.\\
Doch was versteckt hinter einer so großen und komplexen Anwendung? Das folgende Diagramm soll den Ablauf zwischen dem Client und Server verdeutlichen. Daraus wird klar ersichtlich, welche Komponenten beim Anmelden eines Users zum Einsatz kommen. Das System wird folgendermaßen aufgebaut:
\begin{itemize}
\item \textbf{Ebene 1: Client}\\ Unser Anwender, der mittels Browser auf unserer Website surft und seine Mitschriften verwaltet.
\item \textbf{Ebene 2: Web-Server}\\ Nginx nimmt HTTP-Anfragen(GET \& POST) entgegen und überprüft die vom User übermittelten Parameter.
\item \textbf{Ebene 3: Applikations-Server}\\ Hinter Ebene 3 verbirgt sich die eigentliche Geschäftslogik, die maßgebend für unser System ist. Bei DSN werden alle HTTP-Anfragen, die an /api/ gehen, an das Web-Framework Django weitergeleitet. Das ist sozusagen die Schnittstelle zwischen Datenbank und Client.
\newpage
\item \textbf{Ebene 4: Datenbank-Server}\\ Die Datenbank hat die Aufgabe, wichtige, geheimzuhaltende Daten zu persistieren und bei Anfragen schnell Antworten zu liefern.\\
Es muss sich nicht unbedingt um einen Datenbank-Server handeln, sondern es gibt die Möglichkeit, seine Daten auf mehreren Stationen aufzuteilen. Weiters können wichtige von unwichtigen Daten getrennt werden. Das steigert die Performance, sowie die Ausfallsicherheit.
\end{itemize}
\insertpicture{images/design/Ablaufdiagramm.jpg}{Ablaufdiagramm}{(selfmade)}{itm:ablauf-chart}{0.75}

\begin{enumerate}
\item Im ersten Schritt ruft der Anwender mittels einer GET-Anfrage die DSN-Webseite zum ersten Mal auf. Im Hintergrund werden alle notwendigen CSS- und JavaScript-Files geladen.
\item Bevor die Hauptseite erscheint, überprüft der Server die Sprachauswahl. Es besteht die Wahl zwischen zwei Weltsprachen, nämlich Englisch oder Deutsch.\\
Standardmäßig, wird ein POST an \textit{api/change\_lang} mit dem JSON Parameter($\{language:"de"\}$) gesendet.
\item Da es sich um eine /api/ Funktion handelt, leitet der Nginx-Server, der nur für den statischen Teil zuständig ist, die Anfrage weiter zum Django-Server.
\item Hinter der gesendeten Adresse befindet sich eine Funktion, welche den Inhalt der DSN-Seite anhand der übergebenen Parameter auf die gewählte Sprache ändert.
\item Das Ergebnis, die Homepage in der gewählten Sprache, wird an den Client zurückgeliefert.
\item Im nächsten Schritt möchte sich der bestehende DSN-User beim System anmelden. Dafür klickt er auf den Login-Button, wodurch eine GET(/login) Anfrage an dem Server geschickt wird.
\item Mittels REST, Representational State Transfer, wird automatisch zum Login weitergeleitet. Eine erfolgreiche Anmeldung benötigt eine bereits registrierte Email-Adresse, sowie das mindestens 8 stellige Passwort.
\item Durch die Eingabe der Benutzerdaten schickt der Anwender seine Email-Adresse und das verschlüsselte Passwort als JSON-Objekt an den Web-Server. Das POST wird an api/login gesendet.
\item Dynamische Inhalte werden an Django weitergeleitet.
\item Hinter api/login verbirgt sich eine Funktion, die vom JSON-Objekt die Email-Adresse holt. Mit dieser Info stellt der Server bei der Datenbank die Anfrage, ob der User registriert ist und sich rechtmäßig anmelden darf.
\item Die MongoDB-Datenbank, bestehend aus mehreren Schemas, sucht in der Collection \textit{user}. Im Falle eines positiven Ergebnisses, wird der gefundene User an den Web-Server zurückgegeben.
\item Aufgrund des komplizierten Umgangs mit JSON-Objekten wird das Datenbankergebnis mit einem User-Objekt gemappt. Durch den objektorientierten Ansatz kann ohne viel Aufwand das Passwort kontrolliert werden. Sind alle Überprüfungen fehlerlos, wird der User angemeldet.
\item Je nach Response erfolgt die Weiterleitung auf die Management-Page oder es erscheint eine Fehlermeldung.
\end{enumerate}

%\subsubsection{Services}
%Schittstellen
%Services

\newpage

\subsubsection{DSN Architektur}
Unser System setzt sich aus einem Frontend-Server, einem Backend-Server und einer dokumentenbasierten Datenbank zusammen. Das Frontend ist für die Darstellung statischer Daten verantwortlich. Im Gegensatz dazu kümmert sich das Backend mit MongoDB um den dynamischen Teil.
\paragraph{Interaktionen zwischen Frameworks}
\insertpicture{images/design/feinarchitektur.jpg}{Technologische Übersicht}{(selfmade)}{itm:feinarchitektur-chart}{1.0}

Der Client kann über HTTP oder SSL die DSN-Webseite aufrufen. Je nach Anfrage wird eine Verbindung mit dem Frontend- oder Backend-Server aufgebaut.

Der Nginx-Server welcher für das Frontend zuständig ist, beinhaltet statische Daten, wie HTML-Seiten, CSS- \& JS-Dateien und Bilder. Diese werden durch ein JavaScript File namens \textit{routes.js} verwaltet. Eingehende HTTP-Requests werden in diesem File gemappt. Fordert ein User z.B. mit einem GET die Loginpage, wird auf diese weitergeleitet. Jede HTML-Seite hat ihren eigenen Controller, basierend auf JavaScript. Er kümmert sich um die Useraktionen und ändert je nach Anforderung den Inhalt der Website.

\insertpicture{images/design/architektur.jpg}{Aufbau der Server}{(selfmade)}{itm:architektur-chart}{1.0}

Sollte es zu komplexeren Aufgaben kommen, wo statischer Inhalt keine Hilfe ist, wird die Kommunikation mit dem Django-Backend-Server erforderlich. Django ist ein Web Framework basierend auf Python. Wie beim Frontend, existiert ein File, das alles managed. Die \textit{urls.py} Datei nimmt HTTP-Anfragen entgegen und delegiert diese auf die jeweiligen Funktionen im views Ordner. Diese hantieren mit den empfangenen JSON-Objekten. Handelt es sich bei der Anfrage um die Auflistung von Schulheften oder Heftinhalten, ist eine DB-Abfrage erforderlich.\\
Dank dem \textit{models.py} File liefert MongoDB kein JSON, sondern spezifizierte Objekte, die die Arbeit erleichtern.




\newpage

\subsection{Graphische Oberfläche}
\label{subsec:graphischeoberflaeche}
%\section*{Graphische Oberfläche}
\cfoot{Selina Brinnich}

%Text...

\newpage

\subsection{Javascript Optimierung}
\label{subsec:javascriptoptimierung}
%\section*{Javascript Optimierung}
\cfoot{Niklas Hohenwarter}

%Text...

\newpage %----------------------------------------------------------------------------------------------

\section{Implementierung}
\label{sec:implementierung}
%\section*{Umfeldanalyse}
\cfoot{Niklas Hohenwarter}

Dieses Kapitel befasst sich mit der eigentlichen Implementierung des Projektes. Hier beschreibt jedes Teammitglied sein Spezialgebiet. Eine Außnahme stellen hier die Heftelemente dar. Die Implementierung dieser wurde in kleinere Teile aufgeteilt und einzelnen Teammitgliedern zugewiesen.

\subsection{Infrastruktur und Testing}
\label{subsec:infrastrukturtesting}
%\section*{Infrastruktur und Testing}
\cfoot{Niklas Hohenwarter}

\subsubsection{Infrastruktur}
Eine stabile und sichere Infrastruktur und gut getestete Software ist heutzutage ein Muss für jedes IT Projekt. \\
Die Infrastruktur ist wichtig, da in der Vergangenheit oft kleine Projekte bereits wenige Tage nach Veröffentlichung von sehr hohen Userzahlen berichten konnten. Wenn hier zuvor die Infrastruktur gut geplant und implementiert wurde, ist es kein Problem viele User zu bewältigen.\\ 
Ohne Tests wird heute keine Software mehr veröffentlicht, da etwaige Fehler für die Benutzer sehr abschreckend sein können bzw. dem Unternehmen viel Geld kosten können.
\paragraph{Serverhosting}
Die wichtigste technische Grundlage für das Projekt DigitalSchoolNotes ist der Projektserver. Auf diesem Server, wird das Projekt entwickelt und getestet. Hier ist es besonders wichtig, dass das gesamte Team mit der gleichen Umgebung arbeitet, da sonst die einzelnen Codeteile des Teams nicht zusammen funktionieren. Desweiteren wird der Server dazu verwendet, die Zwischenversionen des Projektes öffentlich verfügbar zu machen. Dies ist für das Team essentiell, da dadurch der \gls{Stakeholder} jederzeit Zugriff auf eine aktuelle und stabile Version des Projektes hat. Dadurch kann das Team Änderungswünsche des Stakeholders leichter erfassen und realisieren.\\

Für die Auswahl des Serverhosters wurden einige Kriterien festgelegt. Diese lauten wie folgt:
\begin{itemize}
\item \textbf{Serverstandort:} Der Standort des Projektservers sollte möglichst nahe beim Endbenutzer sein, um die \gls{Latenz} gering zu halten.
\item \textbf{Verfügbarkeit:} Der Server sollte eine hohe Mindestverfügbarkeit haben. Dadurch kann sich der Endbenutzer darauf verlassen, dass das Service erreichbar ist. Der Minimalwert für die Verfügbarkeit wurde auf 99,6\% festgelegt. Das bedeutet, dass der Server für maximal 35h im Jahr nicht verfügbar ist.
\item \textbf{Support:} Der Hoster sollte Support unter der Woche und in Notfällen rund um die Uhr bieten.
\item \textbf{Preis:} Um die Etnwicklungskosten möglichst gering zu halten wurde der maximale Monatspreis auf 10€ festgelegt.
\item \textbf{Wartung:} Der Server sollte sich über ein Webinterface warten lassen.
\end{itemize}

Die oben genannten Kriterien reduzierten die Anzahl der möglichen Hoster stark. Das Team entschied sich für den Hoster netcup GmbH mit sitz in Deutschland. Dieser erfüllte alle Anforderungen und teile des Teams hatten bereits gute Erfahrungen mit dieser Firma gemacht.

Das ausgewählte Produkt der netcup GmbH heißt "Root-Server M v6". Dieser bietet folgende Features:
\begin{itemize}
\item \textbf{Virtualisierungstechnik:}KVM
\item \textbf{CPU:}Intel®Xeon® E5-26xxV3 2,3GHz 2Cores
\item \textbf{RAM:}6GB DDR4
\item \textbf{Speicher:}120GB SSD
\end{itemize}

\paragraph{Erreichbarkeit}
Der Server ist unter der IP-Adresse 37.120.161.195 erreichbar. Da IP-Adressen schwer zu merken sind wurde ebenfalls eine Domain für das Projekt gekauft. Diese Lautet "digitalschoolnotes.com'' und löst auf die oben genannte IP-Adresse auf.

\paragraph{Benutzerverwaltung am Projektserver}
Jedes Projektteam Mitglied hat einen eigenen Unix Account auf dem Projektserver. Der Vorname der Person ist der Benutzername. Das Benutzerpasswort ist von jedem Teammitglied selbst gewählt. Alle Teammitglieder haben sudo rechte. 

\paragraph{Mailsystem}
Das Projektteam hat einen Email-Verteiler mit der Adresse info@digitalschoolnotes.com. Jedes Teammitglied hat eine E-Mail Adresse nach dem Schema des \gls{TGM}s(z.B. nhohenwarter@digitalschoolnotes.com). \\
Der Scrummaster ist unter scrummaster@digitalschoolnotes.com erreichbar.

\paragraph{Serverzugriff}
Um den Server zu konfigurieren und zu verwalten wird mit dem Protokoll SSH darauf zugegriffen. Aus Sicherheitsgründen wurde die Anmeldung mit Passwort verboten und es können hierfür nurnoch SSH Keys verwendet werden. Diese sind um einiges sicherer.
\newpage

\paragraph{Firewall}
Um den Server vor Angriffen und unerwünschten Zugriffen zu schützen wurde eine Firewall installiert. Diese blockiert alle unerwünschten Anfragen. Prinzipiell sind alle Ports geschlossen. Es werden nur Ports geöffnet, welche für das Betreiben des Projektes notwendig sind.\\
Es folgt eine Liste der fregegebenen Ports:
\begin{itemize}
\item 22	SSH
\item 53	DNS
\item 80	HTTP
\item 443	HTTPS
\item 5001-5005 Django Development
\end{itemize}

Die Konfiguration der Firewall des Projektservers sieht wie folgt aus:
\begin{lstlisting}
# Flush the tables to apply changes
iptables -F

# Default policy to drop 'everything' but our output to internet
iptables -P FORWARD DROP
iptables -P INPUT   DROP
iptables -P OUTPUT  ACCEPT

# Allow established connections (the responses to our outgoing traffic)
iptables -A INPUT -m state --state ESTABLISHED,RELATED -j ACCEPT

# Allow local programs that use loopback (Unix sockets)
iptables -A INPUT -s 127.0.0.0/8 -d 127.0.0.0/8 -i lo -j ACCEPT
iptables -A FORWARD -s 127.0.0.0/8 -d 127.0.0.0/8 -i lo -j ACCEPT

#Allowed Ports
iptables -A INPUT -p tcp --dport 22 -m state --state NEW -j ACCEPT
iptables -A INPUT -p tcp --dport 80 -m state --state NEW -j ACCEPT
iptables -A INPUT -p tcp --dport 443 -m state --state NEW -j ACCEPT
iptables -A INPUT -p tcp --dport 53 -m state --state NEW -j ACCEPT
iptables -A INPUT -p udp --dport 53 -m state --state NEW -j ACCEPT
iptables -A INPUT -p tcp --dport 5001 -m state --state NEW -j ACCEPT
iptables -A INPUT -p tcp --dport 5002 -m state --state NEW -j ACCEPT
iptables -A INPUT -p tcp --dport 5003 -m state --state NEW -j ACCEPT
iptables -A INPUT -p tcp --dport 5004 -m state --state NEW -j ACCEPT
iptables -A INPUT -p tcp --dport 5005 -m state --state NEW -j ACCEPT
\end{lstlisting}

Da normalerweise nach einem Reboot des Servers die Firewallkonfiguration verloren geht, musste diese persistiert werden. Das wird durch das Paket \textbf{\textit{iptables-persistent}} erledigt. Die Konfiguration dieses Paketes geschieht wie folgt\cite{FIREWALL_PERSISTENT}:

\begin{lstlisting}
# Install
sudo apt-get install iptables-persistent

# Save Rules
iptables-save > /etc/iptables/rules.v4
\end{lstlisting}

\paragraph{Bruteforce Prevention}
Um Bruteforce Angriffe auf den SSH Dienst zu erschweren wurde am Server fail2ban eingerichtet. Dieses Tool zählt fehlgeschlagene Anmeldeversuche mit und sperrt die IP Adresse des Angreifers nach einer festgelegten Anzahl an Versuchen. Dieses Verfahren ist äußerst effektiv, da der Angreifer dadurch keine Chance hat eine große Anzahl an Passwörtern auszuprobieren(z.B. Wörterbuchangriff). Da auf dem Projektserver die Anmeldung nur mit SSH Key möglich ist, hat der Client welcher sich zum Server verbinden will sechs Versuche einen korrekten SSH Key zu übermitteln.
\paragraph{Webserver}
Als Webserver für unsere Applikation wurde Nginx gewählt. Dieser wurde vor allem gewählt, da das Team bereits in der Vergangenheit mit dieser Software gearbeitet hat. Mithilfe von Nginx kann ebenfalls ein Loadbalancer realisiert werden. Dies ist ein wichtiger Punkt um die Software skalierbar zu halten. \\

Der Webserver ist hauptsächlich für den statischen Content(HTML, Javascript, CSS, Bilder...) zuständig. Die funktioniert indem alle statischen Inhalte in einem Ordner abgelegt werden. Damit weiß Nginx, dass er für diese Inhalte zuständig ist.

\paragraph{SSL}
Um die Daten und Privatsphäre unserer Kunden zu schützen wird bei allen Aufrufen der Website mit SSL verschlüsselt. Um eine legitime SSL Verschlüsselung zu gewährleisten ist ein valides Zertifikat notwendig. Dieses muss von einer Zertifizierungsstelle erworben werden. Das verwendete Zertifikat für das Projekt wurde von der Zertifizierungsstelle namens thawte Inc. erworben. \\

Das Zertifikat validiert die Domain(Domain Validated). Das bedeutet, dass zur Austellung des Zertifikates eine Email an den Besitzer der Domain geschickt wird. Wenn der Besitzer der Doamin der Zertifizierung zustimmt wird diese durchgeführt. \\

Um das Zertifikat nun verwenden zu können muss es mit dem Intermediate Zertifikat der Zertifizierungsstelle verbunden werden. Dadurch ist ein eindeutiger Zertifizierungsfluss hergestellt. Dannach kann es auf den Webserver deployed werden.
   
\paragraph{Produktivbetrieb}
Im Produktivbetrieb ist der Betrieb der Software auf zwei Dienste aufgeteilt. Der statische Teil der Applikation wird wie bereits vorhin beschrieben von Nginx dem User zur Verfügung gestellt. \\

Der dynamische Teil - das Backend - stellt unser Django Server dar. Hier werden die kritischen Operationen wie Datenbankzugriffe oder die Authentifizierung durchgeführt. Wird nun unsere API (der dynamische Teil) aufgerufen, leitet Nginx die Anfrage an den Django Server weiter. Der Django Server läuft mittels gunicorn. Gunicorn ist ein WSGI HTTP Server und somit die Schnittstelle zwischen dem Webserver und Django. Gunicorn startet für Django mehrere Worker Prozesse, wodurch die Anfragen theoretisch auf mehrere CPU Kerne aufgeteilt werden können. Desweiteren startet es die einzelnen Prozesse automatisch neu falls diese abstürzen.
\paragraph{Testbetrieb}
Der Testbetrieb läuft relativ ähnlich wie der Produktivbetrieb ab. Jedes Teammitglied arbeitet an einer eigenen Instanz des Codes. Dadurch behindert sich das Team nicht gegenseitig falls Fehler auftreten. Um dies zu ermöglichen hat jede Person einen eigenen Port zugewiesen bekommen auf der seine Version der Applikation zu erreichen ist. \\

Hat diese Person nun eine Änderung am Code vorgenommen muss dieser auf den Server hochgeladen werden. Nun kann innerhalb der IDE der Django Server gestartet und beendet werden. Dies ist wichtig, da dadurch auch die Fehlermeldungen von Django innerhalb der \gls{IDE} sichtbar sind. Für den statischen Teil ist auch hier Nginx zuständig.

\paragraph{Verfügbarkeit}
Um in Zukunft die Verfügbarkeit zu verbessern gibt es vile Möglichkeiten. Um die Wahrscheinlichkeit eines kritischen Serverausfalles zu reduzieren könnten mehrere Server angemietet werden. Eine andere Möglichkeit wäre es das Hosting in eine Cloud auszulagern(AWS, Google, Azure...). \\

Um die Performance und Verfügbarkeit zu erhöhen sollte ein Loadbalancer verwendet werden. Dieser 

\subsubsection{Testing}
\paragraph{Framework}
\paragraph{Testerstellung}
\paragraph{Probleme}


\newpage

\subsection{User und Rollenmanagement}
\label{subsec:usermanagement}
%\section*{User und Rollenmanagement}
\cfoot{Philipp Adler}

\subsubsection{Usermanagment}
Unter dem Begriff Usermanagement versteht sich, dass verwalten und kontrollieren von Benutzerkonten. Es soll dazu dienen jeden registrierten User eindeutig zu identifizieren und zu kontrollieren, ob die monatlichen Raten für einen Pro-Account überwiesen wurden. Außerdem soll die bereits in Anspruch genommene Speicherkapazität überwacht werden. Dazu ist notwendig, dass jeder User durch eine Kombination von Daten, einmalig, unterscheidbar von anderen ist.
\subsubsection{Authentisierung}
Unter Authentisierung versteht man den Nachweis der behaupteten Identität der BenutzerInnen. Im Falle von DSN handelt es sich hierbei um die eindeutige Email-Adresse, welche einmalig im System benutzt wird. Unter der Identität versteht sich die Sicherheit von wem die Information stammt. Jedes handeln eines Benutzers kann jemanden zugewiesen werden.\\
Ein weiterer Identitätspunkt wäre, dass zu geheim haltenden Passwort, welches aus Sicherheitsgründen mindestens 8 Zeichen beinhalten muss. Durch 8 Zeichen möchten wir Cyberkriminelle das Knacken von Passwörtern erschweren. Für die Abschließung der Registrierung müssen die Nutzungsbedingungen akzipiert werden. \grqq{}Allgemeine Geschäftsbedingungen (AGB) sind vertragliche Klauseln, die zur Standardisierung und Konkretisierung von Massenverträgen dienen. Sie werden von einer Vertragspartei einseitig gestellt und bedürfen daher einer bes. Kontrolle, um ihren Missbrauch zu verhindern.\grqq{}\cite{AGB}\\
\cite{VERTEILTE_SYSTEME}\cite{PASSWORT_SCHUTZ}

\begin{figure}[ht]
\includegraphics[width=0.35\textwidth]{images/usermanagement/Registrierung}
	\caption{Authentisierung bei DSN}
	\label{fig1}
\end{figure}

Um auszuschließen, dass sich eine Software bzw. ein Roboter einen Account auf DSN erstellt, wird ein Captcha verwendet. Ein Captcha dient zur Sicherheit und soll überprüfen wer die Eingabe tätigte.
Um das Captcha anzuzeigen muss eine JS Lib eingebunden werden \cite{CAPTCHA}.
\begin{lstlisting}[caption={Einbindung der JS-Library Recaptcha}]
<script src="https://www.google.com/recaptcha/api.js?
onload=vcRecaptchaApiLoaded&render=explicit" async defer></script>
<script src="https://cdnjs.cloudflare.com/ajax/libs/angular-recaptcha/2.2.5/
angular-recaptcha.min.js"></script>
\end{lstlisting}

Dannach kann das Captcha einfach wie folgt ins Form eingebunden werden:
\begin{lstlisting}
<div vc-recaptcha key="publicKey"></div>
\end{lstlisting}

publicKey definiert den Public Key des Recaptchas. Dieser wird im Javascript file über \$scope zugewiesen.
Das Captcha wird nach der Benutzereingabe serverseitig validiert. Mit den Daten des Registrierungsformular wird ein weiteres Attribut namens recaptcha erhalten. In diesem steht ein Key der zur Validierung an Google gesendet werden muss. Google möchte zur Validierung den Key, die IP des Users und den Secret App Key. Google gibt dann zurück, ob alles korrekt ablief, also ob es sich bei dem User tatsächlich um ein menschliches Lebewesen handelt. Um die Validierung mehrmals einsetzen zu können, haben wir eine Methode dafür geschrieben:

\begin{lstlisting}
def validate_captcha(recaptcha, ip):
    response = {}
    url = "https://www.google.com/recaptcha/api/siteverify"
    params = {
        'secret': settings.RECAPTCHA_SECRET_KEY,
        'response': recaptcha,
        'remoteip': ip
    }
    verify = requests.get(url, params=params, verify=True)
    verify = verify.json()
    response["status"] = verify.get("success", False)
    if response["status"] == True:
        return True
    else:
        return "Captcha ist nicht valide." 
\end{lstlisting}


Wenn das Captcha gelöst wurde, kann man damit das Form genau ein Mal absenden. Im Falle, dass das Formular nochmals abgesendet werden muss, ist es notwendig das Captcha zurück zu setzen.

\begin{figure}[ht]
\includegraphics[width=0.35\textwidth]{images/usermanagement/Captcha}
	\caption{Auslösen des Captchas}
	\label{fig2}
\end{figure}

\subsubsection{Datenmodell}
Da Django-Authentifizierungs Funktionalitäten nur für relative DBMS ausgelegt sind, daher für den vorgesehenen Anwendungszweck nicht geeignet sind, mussten Änderungen vorgenommen werden, um die Authentifizierung über MongoDB zu ermöglichen.\\
Nun muss das User-Model, das eben auf Mongo Basis definiert wurde, noch im File models.py erstellt werden. Der Code für das Model wurde aus dem entsprechenden Source-Code von MongoEngine \cite{MONGOENGINE} kopiert und an unseren Anwendungszweck angepasst.
\begin{lstlisting}
class User(Document):
    id = ObjectIdField(unique=True, required=True, primary_key=True)
    email = EmailField(unique=True, required=True)
    first_name = StringField(max_length=30)
    last_name = StringField(max_length=30)
    password = StringField(max_length=128)
    is_staff = BooleanField(default=False)
    is_prouser = BooleanField(default=False)
    is_active = BooleanField(default=True)
    is_superuser = BooleanField(default=False)
    last_login = DateTimeField(default=datetime.datetime.now())
    date_joined = DateTimeField(default=datetime.datetime.now())
    passwordreset= EmbeddedDocumentField(PasswordReset)
    user_permissions = ListField(ReferenceField(Permission))
    [...]
\end{lstlisting}

Da MongoDB den PrimaryKey als ObjectId verlangt, mussten im bestehenden Original Django-Code folgende Änderungen vorgenommen werden \cite{ISSUE}:\\
Im File \textit{usr/local/lib/python3.4/dist-packages/django/db/models/fields} in Zeile 964: 
\textit{return int(value)}
ändern zu:
\textit{return value}
Im File /usr/local/lib/python3.4/dist-packages/django/contrib/auth in Zeile 111:
\textit{request.session[SESSION\_KEY] = user.\_meta.pk.value\_to\_string(user)}
ändern zu:
\textit{try:
 	request.session[SESSION\_KEY] = user.\_meta.pk.value\_to\_string(user)
except Exception:
 	request.session[SESSION\_KEY] = user.id}

Nach der Authentisierung wird der/die zukünftige BenutzerIn nach dem beschriebenen Datenmodell in die MongoDB hinzugefügt. Da der Registierungsprozess noch nicht abgeschlossen ist, ist das \textit{isactive} Attribute noch auf \textit{False} gesetzt.
Solange das Attribut nicht in einen anderen Zustand übergeht, ist dem/der BenutzerIn nicht erlaubt sich auf DSN anzumelden. Es kann auch sein, dass sich ein bereits registrierter Account sich in diesem Dilemma befinden kann. Das könnte z.B. der Fall sein wenn jemand gegen die Regeln und Gesetze von DSN verstoßt und daraufhin gespeert wird. Dann bleibt ihm nichts anderes übrig Kontakt mit dem Administrator aufzunehmen.  
DATENBANK AUFRUF MAXMUSTERMANN

\subsubsection{Email}
Um den Registrierungsprozess zu beenden, wird dem nahestehenden User ein Token per Mail zugesendet. Dieser dient der Identifizierung und Authentifizierung und könnte folgendermaßen aussehen http://digitalschoolnotes.com/validate/
\\dad9574635aad7d6549536db38f7839c042f7704b3bd74acc427f075d0601470. Bei der Erstellung eines solchen Tokens werden die Email-Adresse des Benutzer, welche als Nachweis dient um zu wissen welcher Account aktiviert werden soll und das aktuelle Datum kombiniert. Diese werden miteinander verknüpft, in einen Hash umgewandelt und in die Datenbank abgespeichert. \textit{"validatetoken" : 
\\"f043ea6e44aea716d08ae2cb70d91bcbb50196da1eb89b4727c124508dbf0d85"}\\
Das Datum dient dazu um dem Token ein Zeitstempel zu geben, dieser bezweckt die Gültigkeit. Wenn dieser Hash nicht in den kommenden 24 Stunden eingelöst wird, verfällt dieser Token, da der Zeitstempel abgelaufen ist und es muss vom User ein neuer angefordert werden. 
Im File authentication/registration.py in Zeile 23 ist die Umsetzung des Registrierungtoken dargestellt:
\begin{lstlisting}
def create_validation_token(email):
    user = User.objects.get(email=email)
    now = datetime.datetime.now()
    to_hash = (str(user.id) + str(now)).encode('utf-8')
    hashed = hashlib.sha256(to_hash).hexdigest()
    hashed = str(hashed)
    user.validatetoken = hashed
    user.save()
    return 'http://digitalschoolnotes.com/validate/' + hashed
\end{lstlisting}

Um eine Email verschicken zu können müssen folgende Konfigurationen unternommen werden. Im \textit{settings.py} muss der Email-Server definiert werden.
\begin{lstlisting}
EMAIL_BACKEND = 'django.core.mail.backends.smtp.EmailBackend'
EMAIL_HOST = 'mxf92d.netcup.net'
EMAIL_HOST_USER = 'noreplyATdigitalschoolnotes.com'
EMAIL_HOST_PASSWORD = 'passwort'
EMAIL_PORT = 25
EMAIL_USE_TLS = False
DEFAULT_FROM_EMAIL = EMAIL_HOST_USER
\end{lstlisting}

\subsubsection{Anmelden}
Wenn der neue Benutzer den Registrierprozess erfolgreich abgeschlossen hat, steht ihm/ihr jetzt frei sich anzumelden. Entweder durch Angabe der Email-Adresse und Passwort oder mittels OAuth. Mit OAuth besteht die Möglichkeit die Registrierung auszulassen und sich direkt anzumelden. OAuth steht für Open Authentication und bietet dem Nutzer die Möglichkeit Daten über einen Webserverice auszutauschen. „OAuth sichert die Programmschnittstelle von Web-Anwendungen ab und verwendet für die Übertragung der Nutzeridentifikation dessen Passwort und einen Token“\cite{OAUTH}. Bei dem Zugriff auf die sensiblen Daten muss der Benutzer keine zusätzlichen Information und auch keine Identität preisgeben. Der Provider holt sich die Benutzerdaten von Facebook oder Google+ und erstellt für den User einen Account.

\begin{figure}[ht]
\includegraphics[width=0.35\textwidth]{images/usermanagement/Anmelden}
	\caption{Klassische Anmeldung oder mittels OAuth}
	\label{fig3}
\end{figure}

Um einen Benutzer anzumelden, muss zunächst ein User-Objekt mit der übergebenen Email Adresse von der Datenbank abgefragt werden. Sollte diese E-Mail Adresse keinem Benutzer zugeordnet sein, existiert der Benutzer noch nicht. Ansonsten muss mit \textit{.check\_password(...)} das eingegebene Passwort überprüft werden. Sollte dieses korrekt sein, kann der User angemeldet werden. Dazu muss das Authentication-Backend und ein Session Timeout gesetzt werden und der Benutzer über die Funktion \textit{login()} eingeloggt werden:
\begin{lstlisting}
try:
    user = User.objects.get(email='exampleATexample.com')
except:
    user = None
if user is not None and user.check_password('myPassword'):
    user.backend = 'mongoengine.django.auth.MongoEngineBackend'
    login(request, user)
    request.session.set_expiry(60 * 60 * 1) # 1 hour timeout
\end{lstlisting}

Der aktuell angemeldete Benutzer kann mittels \textit{request.user} abgefragt werden. Sollte der Benutzer aktuell nicht angemeldet sein, ist dies null, ansonsten wird das entsprechende User-Objekt zurückgeliefert.
Um einen angemeldeten Benutzer wieder abzumelden muss lediglich folgende Funktion ausgeführt werden:
\textit{logout(request)}

\subsubsection{Usersicht}

\begin{figure}[ht]
\includegraphics[width=0.85\textwidth]{images/usermanagement/Usersicht}
	\caption{Navigationsleiste als User}
	\label{fig4}
\end{figure}

Ein angemeldeter Benutzer kann seine Benutzerinformationen nachgiebig unter Kontoeinstellungen ändern. Für die Änderung seiner Daten, muss aus Sicherheitsgründen, dass aktuelle Passwort eingegeben werden.

\begin{figure}[ht]
\includegraphics[width=0.35\textwidth]{images/usermanagement/Kontoeinstellungen}
	\caption{Kontoeinstellungen}
	\label{fig5}
\end{figure}

Im System befinden sich drei verschiedene Berechtigungsstufen, welche sind: der Standard-Benutzer, Pro-Benutzer und Administrator. Jedem/r registrierten AnwenderIn ist zu Beginn ein Standard-Benutzer. Ihnen stehen eine begrenzte Anzahl an digitalen Heften zur Verfügung. Durch eine geringe monatliche Zahlung kann der Standard-Account zum Pro-Account upgegradet werden, wodurch dem/r SchülerIn erweiterte Funktion angeboten werden. Zum einen stehen mehr Hefte zur Verfügung, es wird keine Werbung angezeigt, sowie keine Speicherbeschränkung. Die letzte Berechtigungsstufe sind Administratoren. Sie sind ebenfalls Pro-User, haben im Gegensatz einen eigenen Admin-Bereich, wo sämtliche Daten über Benutzer verwaltet und kontrolliert werden können. Dieser Bereich kann mit /admin nach der URL aufgerufen werden. Er unterscheidet sich durch den schwarzen Menübalken.\\
Jede/r BenutzerIn hat die Möglichkeit nach seinen Freunden oder anderen registrierten Anwendern zu suchen. Rechts oben in der Navigationleiste befindet sich ein Suchbalken, der es Anwendern ermöglicht nach Vornamen, Nachnamen oder nach Email-Adressen zu suchen.

\begin{figure}[ht]
\includegraphics[width=0.85\textwidth]{images/usermanagement/Search}
	\caption{Usersuche}
	\label{fig6}
\end{figure}

Bei Klick auf den gefundenen User gelangt auf dessen Profil. Auf dem Profil ist zu einem der volle Name zu sehen, die Email-Adresse und die Berechtigungsstufe, ob es sich um einen Standardbenutzer, Pro-Benutzer oder Administrator handelt. Ein weiterer wichtiger Punkt sind die öffentlichen Hefte. Jeder User ist in der Lage von öffentlichen Heften andere einzelne Seiten zu exportieren.

\subsubsection{Adminsicht}

\includegraphics[width=0.35\textwidth]{images/usermanagement/Adminsicht}\\

Auf der User Management Page werden alle Benutzer von DSN aufgelistet. Man hat Einsicht auf die Email-Adresse, Vorname, Nachname und auf die Berechtigungsstufe, Standard-Benutzer, Pro-Benutzer, Administrator. Außerdem besteht die Möglichkeit als Administrator andere Benutzer zu löschen, die Berechtigungsstufe zu ändern oder den Benutzer mittels einer Mail auf etwas hinzuweisen. Neben der Auflistung der Benutzer, kann auch nach einer bestimmten Person suchen. Die Eingabe wird mit den Vorname, Nachnamen und der Email-Adresse verglichen. Mittels \textit{collectionName.objects()} liefert Mongoengine alle Objekte von der angegeben Collection. Wenn der Adminstrator aber nicht alle Objekte von einer Collection haben möchte, sondern nur eine gewisse Anzahl, können diese mit folgenden Befehl abgefragt werden:\textit{Tabellenname.objects[x:y]}\\
Objekte können wie folgt gesucht werden:\\
\textit{users(Q(email\_\_icontains=suchtext) \big| Q(first\_name\_\_icontains=suchtext) \big|\\
Q(last\_name\_\_icontains=suchtext))}\\
\\
Objekte können auch nach einer bestimmten Spalte sortiert werden.\\
\textit{users.order\_by(spaltenname)
users.order\_by('- '+spaltenname)}\\
\\
Falls man ein vorhandenes Objekt aus der Collection löschen möchte, muss dieses zuvor rausfiltern und dann mit der Funktion \textit{delete()} entfernen. Veränderte Daten werden mit der Funktion \textit{save()} persistiert.

\includegraphics[width=0.65\textwidth]{images/usermanagement/Usermanagment}\\

Da unser User Management Page nur Seitenweise die Benutzerliefert, um Ladezeiten zu minimieren, muss diese mittels Pagination umgesetzt werden. Für die Realisierung sind die Anzahl der Objekte die insgesamt ausgegeben werden sollen gefordert. Dann wird definiert, wieviele Elemente pro Seite angezeigt werden sollen. Um die Seitenanzahl zu berechnen,  werden alle Elemente durch die Anzahl der Elemente dividiert, die pro Seite dargestellt werden sollen. Der Server übergibt dann die Elemente, welche auf einer Seite angezeigt werden sollen. Falls der User auf eine andere Seite wechselt, werden die nächsten Objekte vom Server bezogen.
\begin{lstlisting}
$http({
        method: 'GET',
        url: '/api/admin_user',
        data: {}
    })
        .success(function (data) {
            $scope.users = data['test'];
            $scope.len = data['len'];
            $scope.currentPage = 0;
            $scope.l = Math.ceil($scope.len/$scope.itemsPerPage);
        })
        .error(function (data) {
  });
    var searchMatch = function (haystack, needle) {
        if (!needle) {
            return true;
        }
        return haystack.toLowerCase().indexOf(needle.toLowerCase()) !== -1;
    };

    $scope.range = function (size, start, end) {
        var ret = [];
        if (size < end) {
            end = size;
            start = size;
        }
        for (var i = start; i < end; i++) {
            ret.push(i);
        }
        return ret;
    };

    $scope.firstPage = function () {
        $scope.currentPage = 0;
    };

    $scope.prevPage = function () {
        if ($scope.currentPage > 0) {
            $scope.currentPage--;
        }
    };

    $scope.nextPage = function () {
        if ($scope.currentPage < $scope.l- 1) {
            $scope.currentPage++;
        }
    };

    $scope.lastPage = function () {
        $scope.currentPage = $scope.l-1;
    };

    $scope.setPage = function () {
        $scope.currentPage = this.n;
    };

\end{lstlisting}

\begin{lstlisting}
def view_users(request):
    if not request.user.is_authenticated() or not request.user.is_superuser:
        return JsonResponse({})
    u = []
    length = 0
    weiter = False
    delete = False
    if request.method == "GET":
        users = User.objects[0:20]
        length = len(User.objects)
    elif request.method == "POST":
        params = json.loads(request.body.decode('utf-8'))
        von = (params['Page']-1)*params['counter']
        bis = params['counter']*params['Page']

        try:
            """ Delete """
            user = User.objects.get(email=params['email'])

            if user != request.user and user.delete_date == None:
                enddate = datetime.now() + timedelta(days=7)
                until = date(enddate.year, enddate.month, enddate.day)
                user.delete_date = until
                user.save()
                deleteemail(user.email, user.first_name, until)
            elif user != request.user:
                user.delete_date = None
                user.save()
        except KeyError:
            pass

        users = User.objects()

        try:
            """ Search """
            if bool(params['text'] and params['text'].strip()):
                users = users(Q(email__icontains=params['text']) | Q(first_name__icontains=params['text']) | Q(last_name__icontains=params['text']))
        except KeyError:
            pass

        try:
            """ Sort """
            if params['order'] is not None:
                if params['order']:
                    users = users.order_by(params['spalte'])
                else:
                    users = users.order_by('-'+str(params['spalte']))
        except KeyError:
            pass
        length = len(users)
        users = users[von:bis]
    for user in users:
        security = 1
        if user.is_prouser: security = 2
        if user.is_superuser: security = 3
        if not user.is_active: security = 4
        if user.delete_date == None:
            delete_state = 'Account loeschen'
        else:
            days = abs(datetime.today().day - int(date.strftime(user.delete_date, "%d")))
            delete_state = ' Loeschung in %s Tagen' % (str(days))

        u.append({
            "email": user.email,
            "first_name": user.first_name,
            "last_name": user.last_name,
            "security_level": security,
            "delete_account": delete_state
        })
    if length == 0:
        return JsonResponse({'test': u})
    else:
        return JsonResponse({'test': u, 'len': length})
\end{lstlisting}

Falls ein Pro-User seinen Zahlungen nicht nachkommt oder sich durch Böswilligkeiten bemerkbar macht hat der Administrator von DSN das Recht diesen User zu löschen. DSN gibt den User die Möglichkeit seine Daten bzw. Hefte bevor er gelöscht wird zu sichern. Der zu löschende Benutzer empfängt eine Email, wo darauf hingewiesen wird das sein Account und alle dazugehörigen Daten nach 7 Tagen gelöscht werden. Am Server von DSN läuft ein cronjob welcher jeden Tag das definierte Command, welches überprüft wann der zu löschende User entfernt werden soll, ausführt. Im Falle das jemand länger als 3 Monate interaktiv ist, wird ihm eine Informationsmail zugesendet. Diese informiert ihn, dass er sich in den 7 kommenden Tagen einloggen soll, ansonsten wird der Account mit alle den Daten gelöscht. \cite{COMMANDS}\cite{CRON}

\begin{lstlisting}
from django.core.management import BaseCommand
from datetime import *
from dsn.models import User
from dsn.authentication.account_delete import delete_account

#https://docs.djangoproject.com/en/1.9/howto/custom-management-commands/
#The class must be named Command, and subclass BaseCommand
class Command(BaseCommand):
    # Show this when the user types help
    help = "Command for the User notification"

    # A command must define handle()
    def handle(self, *args, **options):
        until = datetime.now() + timedelta(days=7)
        users = User.objects(delete_date__lte=until)
        for user in users:
            now = datetime.today()
            day = abs(now.day - int(date.strftime(user.delete_date, "%d")))
            if day == 0:#User delete
                delete_account(user)
\end{lstlisting}

Cronjob
\begin{lstlisting}
# m h  dom mon dow   command
# * *   */1   *   *    python3 /home/stable/dsn/manage.py inform
# * *   */1   *   *    python3 /home/stable/dsn/manage.py delete
\end{lstlisting}

Unter dem Navigationspunkt Bills werden die Rechnungen von den Pro-Benutzern aufgelistet. Zum einen wann und ob der Betrag eingezahlt wurde.

\newpage

\subsection{Datenmanagement}
\label{subsec:datenmanagement}
%\section*{Datenmanagement}
\cfoot{Selina Brinnich}

Ein gutes Datenmanagement ist eine Grundvoraussetzung für eine
gut funktionierende Applikation. Alle Daten müssen persistiert werden, um nicht verloren zu gehen. Zudem soll durch ein gut organisiertes Datenmanagement eine einfache und effiziente Verwaltung der Daten gesichert werden.

\subsubsection{Persistierung von Daten}
Die Persistierung von Daten ist besonders wichtig. Das Abspeichern aller Daten dient dem Zweck, die Daten auch zu einem späteren Zeitpunkt oder beispielsweise nach einem Reboot des Servers noch abrufen zu können. Um das zu erreichen, wird eine Datenbank benötigt, die sich um die Persistierung kümmert. Dabei wird zwischen zwei Arten von Datenbankkonzepten unterschieden: Relationale Datenbanken und NoSQL Datenbanken. Im folgenden werden beide Konzepte näher erläutert und ein Vergleich zwischen den beiden Arten erstellt.
\paragraph{Relationale Datenbanken}
% Erklärung Relationale DB...
% Vorteile + Nachteile
\paragraph{NoSQL Datenbanken}
% Erklärung NoSQL DB...
% Vorteile + Nachteile
\paragraph{Vergleich}
% Was war uns wichtig...
% Warum NoSQL, warum MongoDB...


\subsubsection{Datenmodell}
Um einen einheitlichen Zugriff auf Daten zu ermöglichen, muss ein Datenmodell entwickelt werden. Ein Datenmodell bezeichnet eine Struktur beziehungsweise den Aufbau der einzelnen Datensätze. Dabei ist zu beachten, dass in der Applikation Daten unterschiedlichster Art gespeichert werden. Das bedeutet, dass diese unterschiedlichen Typen logischerweise auch unterschiedliche Datenmodelle zugrunde haben müssen. Zur Organisation dieser unterschiedlichen Arten von Daten gibt es in MongoDB sogenannte Collections. Diese Collections beinhalten alle Datensätze eines Typs, beispielsweise alle Einträge von Benutzern. In unserer Applikation gibt es folgende Collections:
\begin{itemize}
\item \textbf{user}\\ Zur Abspeicherung von Benutzeraccounts
\item \textbf{notebook}\\ Zur Abspeicherung von Schulheften; beinhaltet auch alle Daten innerhalb des Heftes (Texte, Bilder,...)
\item \textbf{time\_table}\\ Zur Abspeicherung eines Stundenplans; beinhaltet die einzelnen Stundenzeiten sowie eine Fachbezeichnung, einen Lehrer und einen Raum pro Tag und Stunde
\end{itemize}

\paragraph{User}
Die Collection \textit{user} beinhaltet alle relevanten Daten zur Abspeicherung von Benutzeraccounts. Folgendes Datenmodell liegt dem zugrunde:
\begin{itemize}
\item \textbf{\_id}\\ Eine Object-ID zur eindeutigen Identifizierung eines Eintrags
\item \textbf{email}\\ Die E-Mail Adresse des jeweiligen Benutzers
\item \textbf{first\_name}\\ Der Vorname des jeweiligen Benutzers
\item \textbf{last\_name}\\ Der Nachname des jeweiligen Benutzers
\item \textbf{password}\\ Ein Hash eines selbst definierten Passwortes des jeweiligen Benutzers
\item \textbf{is\_prouser}\\ Beschreibt, ob der jeweilige Benutzer einen Pro-Account besitzt
\item \textbf{is\_active}\\ Beschreibt, ob der jeweilige Benutzer bereits seine E-Mail Adresse bestätigt hat
\item \textbf{is\_superuser}\\ Beschreibt, ob der jeweilige Benutzer ein Administrator der Applikation ist
\item \textbf{last\_login}\\ Das Datum, an dem der jeweilige Benutzer sich das letzte Mal erfolgreich an der Applikation angemeldet hat
\item \textbf{date\_joined}\\ Das Datum, an dem der jeweilige Benutzer sich an der Applikation registriert hat
\item \textbf{validatetoken}\\ Ein Hash, der als Token zur Bestätigung der E-Mail Adresse des jeweiligen Benutzers dient, sofern dieser die Bestätigung noch nicht durchgeführt hat
\item \textbf{passwordreset}\\ Ein Hash sowie ein Datum, welche zur Zurücksetzung des Passwortes des jeweiligen Benutzers dienen, sofern dieser angegeben hat, sein Passwort vergessen zu haben
\end{itemize}
Benutzer werden innerhalb der Applikation mithilfe ihrer E-Mail Adresse identifiziert. Das bedeutet, dass die Eigenschaft \textit{email} bei jedem Eintrag in der Datenbank einzigartig sein muss, ebenso wie die ID des Eintrags.

\paragraph{Notebook}
Die Collection \textit{notebook} beinhaltet die Daten, die ein einzelnes Heft betreffen. Dazu wird folgende Struktur definiert:
\begin{itemize}
\item \textbf{\_id}\\ Eine Object-ID zur eindeutigen Identifizierung eines Eintrags
\item \textbf{name}\\ Der Anzeigename des Heftes, der vom Benutzer festgelegt wurde
\item \textbf{is\_public}\\ Beschreibt, ob das Heft öffentlich (von allen Benutzern der Applikation) einsehbar ist
\item \textbf{create\_date}\\ Das Erstellungsdatum des Heftes
\item \textbf{last\_change}\\ Das Datum, an dem das Heft das letzte Mal bearbeitet wurde
\item \textbf{email}\\ Die E-Mail Adresse des Besitzers des Heftes
\item \textbf{numpages}\\ Die Anzahl an Seiten, die das Heft besitzt
\item \textbf{current\_page}\\ Die Seite, die aufgeschlagen wird, sobald der Benutzer das Heft das nächste Mal öffnet
\item \textbf{content}\\ Eine Liste, die alle Inhalte des Heftes (Texte, Bilder,...) beinhaltet
\item \textbf{collaborator} Eine Liste an E-Mail Adressen von Benutzern der Applikation, die neben dem Besitzer des Heftes ebenfalls die Inhalte des Heftes bearbeiten dürfen
\end{itemize}
Der Name eines Heftes ist pro Benutzer einzigartig zu vergeben, um das Heft identifizieren zu können. Abgesehen von der Eigenschaft \textit{\_id} kann ein Heft also auch mithilfe der beiden Eigenschaften \textit{name} und \textit{email} eindeutig identifiziert werden. Die Eigenschaft \textit{content} besteht aus einer Liste. Diese Liste beinhaltet individuell viele JSON-Objekte, die jeweils ein Objekt innerhalb des Heftes darstellen (Text, Bild,...). Diese JSON-Objekte bestehen wiederum aus einer ID zur Identifizierung, der Art des Elementes (Textelement, Bildelement,...), der genauen Position innerhalb des Heftes (Seitenzahl und x-Koordinate, sowie y-Koordinate auf dieser Seite) und dem eigentlichen Inhalt, beispielsweise dem Text den der Benutzer eingegeben hat, sollte es sich um ein Textelement handeln.

\paragraph{Timetable}
In der Collection \textit{time\_table} werden alle Daten der Stundenpläne von Benutzern gespeichert. Das beinhaltet sowohl die einzelnen Fächer, Lehrer und Räume pro Stunde, als auch die per Benutzer definierten Zeiten für jede Stunde. Zur Persistierung wird folgendes Datenmodell verwendet:
\begin{itemize}
\item \textbf{\_id}\\ Eine Object-ID zur eindeutigen Identifizierung eines Eintrags
\item \textbf{email}\\ Die E-Mail Adresse des Benutzers, dem der Stundenplan zugeordnet ist
\item \textbf{times}\\ Eine Liste, die die Anfangs- und Endzeiten jeder Stunde im Stundenplan enthält
\item \textbf{fields}\\ Eine Liste, die alle einzelnen Stunden im Stundenplan mit Fach, Lehrer und Raum enthält
\end{itemize}
Der Stundenplan wird einem Benutzer mithilfe der Eigenschaft \textit{email} zugeordnet und damit auch eindeutig identifiziert, da jeder Benutzer nur einen Stundenplan besitzen kann. \\
Die Eigenschaften \textit{times} und \textit{fields} sind jeweils Listen, die mehrere JSON-Objekte enthalten.Ein JSON-Objekt in der Liste \textit{times} enthält die jeweilige Stunde (1-xx) und die Anfangs- und Endzeit für diese Stunde. Ein JSON-Objekt in der Liste \textit{fields} enthält eine ID zum Identifizieren der jeweiligen Stunde (Zusammengesetzt aus Reihenzahl und Spaltenzahl im Stundenplan), eine Bezeichnung des Faches, das in dieser Stunde unterrichtet wird, dem Namen des Lehrers, der das jeweilige Fach unterrichtet und dem Raum, in dem der Unterricht stattfindet, sowie ein Heftname eines Heftes, das dem jeweiligen Benutzer gehört. Mithilfe des Heftnamens kann ein Heft mit einer Stunde im Stundenplan verknüpft und zugeordnet werden.

\subsubsection{Datenzugriff}
Der Zugriff auf Daten in der Datenbank kann über zwei Arten erfolgen. Entweder es wird direkt über die Konsole von MongoDB zugegriffen, oder der Datenzugriff erfolgt über die Applikation. \\
Der Zugriff auf Daten bezeichnet dabei unterschiedliche Operationen. Es können neue Daten erstellt, vorhandene Daten ausgelesen und bestehende Daten bearbeitet oder gelöscht werden.
\paragraph{Direkter Zugriff auf die Datenbank}
\paragraph{Zugriff aus der Applikation}

\newpage

\subsection{Heftelemente}
\label{subsec:heftelement}
%\subsection*{Heftelemente}
\cfoot{}
Um Inhalt in ein Heft einzufügen gibt, es die Ebene der Elemente. Diese verschiedenen Elemente bauen alle auf der selben Grundstruktur auf und sind nach Art des Elements weiter spezifiziert.
\insertpicture{images/elemente/elemente}{Heftelemente}{(selfmade)}{itm:elemente}{0.8}
Mittels Mausklick auf ein bestimmtes Icon auf der Toolbar wird das jeweilige Element in das Heft eingefügt.

Um das Element bearbeiten zu können, wird beim Überfahren des Elements ein Bleistift eingeblendet, welcher dem Benutzer erlaubt, den bestehenden Inhalt des Elements zu editieren. Neben dem Bleistift erscheint ebenfalls ein Mistkübel, mit dem das Element entfernt werden kann.
Außerdem ist es möglich, ein Element beliebig im Heft zu verschieben und dadurch beliebig im Heft zu platzieren. 

Diese Grundfunktionen sind in jedem Element enthalten. Jedes Element ist für den jeweiligen Anwendungsbereich genauer spezialisiert.

\subsubsection{Textelement}
\label{subsubsec:textelement}
%\subsubsection*{Textelement}
\cfoot{Thomas Stedronsky}

%Text...

\newpage

\subsubsection{Codeelement}
\label{subsubsec:codeelement}
%\subsubsection*{Codeelement}
\cfoot{Adin Karic}

%Text...

\newpage

\subsubsection{Bildelement}
\label{subsubsec:bildelement}
%\subsubsection*{Bildelement}
\cfoot{Niklas Hohenwarter}

%Text...

\newpage

\subsection{Parallel Working System}
\label{subsec:parallelws}
%\section*{Parallel Working System}
\cfoot{Thomas Stedronsky}
Mit dem Parallel Working System ist es dem Benutzer möglich Hefte mit anderen Benutzern zu teilen und diese anschließend gleichzeitig zu bearbeiten. Es kann immer ein Element pro User bearbeitet werden. Während dieses Element bearbeitet wird können anderen Benutzer dieses Element weder löschen, bearbeiten oder verschieben. 
\subsubsection{Bestehende Systeme}
\paragraph{Google Docs}
\paragraph{Microsoft Portal}
\subsubsection{Umsetzung}
Das Parallel Working System besteht aus zwei Module. Mit dem ersten Modul ist es möglich Hefte mit anderen Nutzern zu teilen. Das zweite integrierte Modul ist für die ständige Aktualisierung und Synchronisation des Heftinhaltes verantwortlich. Wobei das zweite Modul deutlich größer ist.
\paragraph{Hefte teilen}
Jedes Heft hat ein Attribut \textit{collaborator}(siehe 5.3.2.2). Dieses Attribut ist die Vorraussetzung um das Heft mit anderen Benutzern zu teilen. 
Um anderen Nutzern ein bestimmtes Heft freizugeben gibt es folgende Oberfläche:
\insertpicture{images/pws/add_collaborator}{Hinzufügen eines weiteren Nutzers}{(selfmade)}{itm:collaborator-chart}{0.70}
Wie man in der Abbildung erkennen kann, werden außerdem Vorschläge von E-Mail Adressen gegeben, um den User den Umgang mit dem PWS zu erleichtern.\\

Ist ein Heft nun für einen User freigegeben, so wird das Heft in dessen Heftkollektion unter 'Für mich freigegebene Hefte' angezeigt. \\
Um die Teilung eines Heftes aufzuheben gibt es zwei Möglichkeiten. Entweder der Besitzer des Heftes löscht den User als \textit{collaborator} mittels dem roten Minus neben den Namen (siehe Abbildung 11). Die andere Möglichkeit ist, dass der geteilte User sich selbst die Rechte entzieht.
\insertpicture{images/pws/show_notebook}{Für mich freigegeben Hefte}{(selfmade)}{itm:showNotebook-chart}{0.70} 

Der Benutzer denn ein Heft freigegeben wurde hat nur die Bearbeitungsrechte, er kann weder den Namen ändern noch das Heft löschen. Wird das Notebook vom Besitzer gelöscht, dann wird das Heft auch bei allen Mitbesitzern gelöscht. Hefte können nicht übergeben werden.
\paragraph{Aktualisierung und Synchronisation}
\subsubsection{Probleme}
\subsubsection{Ausblick}


%Text...

\newpage

\subsection{Optical Character Recognition}
\label{subsec:ocr}
%\section*{Optical Character Recognition}
\cfoot{Adin Karic}
Optical Character Recognition oder auch Optische Zeichenerkennung bezeichnet die automatisierte Texterkennung innerhalb von Bildern. Mit diesem Verfahren können also Bilder eingelesen werden und die darin enthaltenen Zeichen als diese erkannt und ausgegeben werden. OCR ermöglicht es den Nutzern von Digital School Notes Bilder auf den Server zu laden, den textuellen Inhalt aus diesen Bildern zu filtern und als bearbeitbares Textelement in ein Heft einzufügen.

\insertpicture{images/ocr/ocrbsp.png}{Beispiel einer Texterkennung}{\cite{OCRBSP}}{itm:ocrbsp}{0.7}

Die ersten Versuche einer automatischen Zeichenerkennung lassen sich auf die 30er Jahre datieren. Die damals entworfene Maschine für das künstliche Lesen konnte mit einem Mustervergleich die zehn arabischen Ziffern automatisch erkennen. Solche Texterkennungssysteme konnten sich vor allem sehr früh bei Banken und Versicherungen etablieren. Bereits zu Beginn der 60er Jahre war es möglich, mit entsprechenden Rechnern, geschriebene oder gedruckte Zeichen auf Belegen zu identifizieren.

In den folgenden Jahren gelang dann aber auch der Durchbruch beim Erkennen von Buchstaben. Dabei war die Anzahl der benötigten Schriftarten jedoch sehr begrenzt. Die erste Schriftfamilie, OCR-A, beinhaltete neben Ziffern und Großbuchstaben auch 30 Sonderzeichen. Die Weiterentwicklung davon, OCR-B, ermöglichte auch die Erkennung von Kleinbuchstaben.

In den 70er Jahren haben sich bei Banken und Versicherungen schließlich intelligentere Systeme etabliert, die neben ungenormten Schriftarten auch Handschriften erkennen konnten. Da diese Lesegeräte Komplettsysteme waren, also aus Hardware und Software bestanden, waren die Preise dementsprechend hoch.

Erst seit Mitte der 80er Jahre etablierten sich, aufgrund der Entwicklung von Flachbett- und anderen Scannern, sowie leistungsfähigeren Computern, billigere Systeme, die gute Lösungen anboten. Diese OCR-Lösungen haben sich im Laufe der Jahrzehnte stetig weiterentwickelt, bis zu der professionellen Software, die man heute im Handel oder frei verfügbar im Internet findet.

\subsubsection{OCR-Verfahren}
Das Verfahren zur optischen Zeichenerkennung teilt sich grob gesagt in drei Teile:
\begin{itemize}
\item Seiten- und Gliederungserkennung
\item Mustererkennung
\item Umwandlung in das Ausgabeformat
\end{itemize}
Zudem gibt es verschiedene Arten der konkreten Zeichenerkennung.

\paragraph{Bitmustervergleich}
Ein besonders einfaches und wahrscheinlich das älteste Verfahren zur Zeichenerkennung ist der Bitmustervergleich (Matrizenvergleich). Das in den 30er Jahren entwickelte Verfahren vergleicht Zeichen miteinander, indem es die Differenz des digitalisierten Zeichens zu den gespeicherten Zeichenmustern betrachtet. Wenn ein betrachtetes Zeichen eine geringe Differenz zum gespeicherten Muster zeigt, wird es als dieses Zeichen identifiziert. Für jedes Zeichen existiert also eine Matrix, in der dieses Zeichen in ein Pixelmuster zerlegt wird.

\insertpicture{images/ocr/bitmuster.jpg}{Bitmustervergleich}{\cite{OCRB}}{itm:bitm}{0.7}

Bei diesem Vergleich wird das spezifische Pixelmuster des gescannten Zeichens nacheinander mit den vorhandenen Mustern in einer Datenbank verglichen. Dadurch wird durch die Übereinstimmung der Bildpunkte eine Korrelation errechnet. Die Schwierigkeiten bei diesem Verfahren bestehen darin, dass die beiden zu vergleichenden Bitmuster selten wirklich identisch sind. Das liegt einerseits an der Qualität des gescannten Zeichens, welche von Helligkeit und Kontrast abhängig ist. Andererseits könnten beim Scanvorgang selbst einige Verzerrungen oder Verschmutzungen auftreten. Die Größenunterschiede können zusätzlich eine wesentliche Rolle spielen, da die Muster zunächst auf eine einheitliche Größe angepasst werden müssen. Bei diesem Vorgang können wichtige Detailinformationen verloren gehen.

Um diesen Problemen aus dem Weg zu gehen, versucht man mit Hilfsmitteln, sogenannten Normierungsverfahren, die Konturen der Zeichen zu begradigen. So könnten Zeichen, die auch geringe Abweichungen zu den Musterzeichen aufweisen, erkannt werden. Die erzielte Genauigkeit beim Erkennen der Zeichen ist ein wesentlicher Faktor zur Beurteilung der verschiedenen Verfahren. Beim Bitmustervergleich ist die Erkennungsgenauigkeit einfach zu niedrig. Aufgrund dieser Tatsache hat sich das Verfahren auch nicht durchsetzen können. Ein Bitmustervergleich ist nur dann vorteilhaft, wenn die Vorlagen eine sehr gute Qualität haben und wenige Schriftarten benutzt wurden. Unter diesen Voraussetzungen ist ein Bitmustervergleich meist schneller als die anderen Methoden.

\paragraph{Klassifikation}
Bei der Klassifikation handelt es sich um ein komplexeres, aber auch flexibleres Verfahren. Die Zeichen werden hier anhand sogenannter Klassen analysiert. Es werden nicht alle Pixel der Zeichen, sondern nur spezifische Merkmale untersucht, die wiederum in einzelne Klassen unterteilt werden. Die Entwickler des Verfahrens können bestimmen, nach welchen Klassen unterteilt wird und wie viele Klassen es überhaupt gibt.

\insertpicture{images/ocr/klassi.jpg}{Klassenbaum für Merkmalsanalyse}{\cite{OCRB}}{itm:klassi}{1.0}

In einem Klassenbaum werden die Zeichen systematisch anhand ihrer Merkmale in Klassen unterteilt. Innerhalb der Klassen muss dann wieder eine weitere Klassifikation stattfinden, um am Ende des Prozesses ein eindeutiges Zeichen zuzuordnen. Es kann auch durchaus vorkommen, dass verschiedene Zeichen in mehreren Klassen vorkommen. Typische Kriterien für die Klassenbildung sind zum Beispiel die Anzahl der Zyklen, die Anzahl der Teile und die Eindringtiefe (Größe der Zeichen).

Diese Merkmalsanalyse liefert zu jedem Zeichen einen Merkmalsvektor mit entsprechenden Merkmalseinträgen. Den Merkmalen werden also konkrete Werte, die diese Merkmale beschreiben, zugeordnet. Sobald ein solcher Vektor mit den Merkmalen eines schon bekannten Zeichens übereinstimmt, kann das Zeichen identifiziert werden.

Durch diese Umstände reagiert dieses Verfahren viel flexibler auf verschiedene Schriftarten. Nur bei sehr starken Abweichungen der Merkmale kann es zu Schwierigkeiten bei der Analyse kommen. Die Auswahl der richtigen Klassenmerkmale und der Anzahl der nötigen Klassen ist entscheidend für dieses Verfahren. Bei zu starken Unterschieden im Schriftbild kann es jedoch zur Überforderung dieser Methode kommen.

\paragraph{Topologische Analyse}

Diese Methode wird auch strukturelle Formenanalyse genannt und orientiert sich, ebenso wie die Klassifikation, an spezifischen Eigenschaften der Zeichen und nicht an umfangreichen Mustervergleichen.

Der erste Schritt bei der topologischen Analyse ist die Reduzierung des Pixelbilds auf dessen Grundelemente. Dieser Vorgang wird Vektorisierung genannt. Die überflüssigen Pixel werden also alle entfernt und es entsteht ein ,,Zeichenskelett". Danach folgt die eigentliche Zeichenanalyse. Das Verfahren sucht nun nach Beschreibungen, die mit den Einzelelementen des reduzierten Zeichens übereinstimmen. 

\insertpicture{images/ocr/vektor.jpg}{Vektorisierung eines Zeichens}{\cite{OCRB}}{itm:vektor}{0.5}

Diese Methode ermöglicht eine sehr fehlertolerante Zeichenerkennung. Sie benutzt komplexe Algorithmen, konzentriert sich aber nur auf wenige Elemente und erreicht dadurch eine angemessene Geschwindigkeit.

\paragraph{Neuronale Netze}
Neuronale Netze sollen die Schwächen der bereits genannten Verfahren beheben. Mit solchen neuronalen Netzen wird versucht, intelligentes Verhalten auf Computern nachzuvollziehen. Dabei stellt die Neuroinformatik eine Alternative zur klassischen algorithmischen Informatik dar. Die parallele und verteilte Informationsverarbeitung selbst geschieht mit Hilfe einer Vielzahl von Elementen, den sogenannten Neuronen.

Eine wesentliche Eigenschaft bei diesem Verfahren ist die Selbstprogrammierung. Das System kann sein Wissen durch selbstständiges Lernen erweitern. Das neuronale Netz muss aber zunächst kalibriert werden, dies geschieht beispielsweise mit einem Training aus Identifikationsmustern. Sobald das System ein vorliegendes Problem dann mit einem vernachlässigbaren Fehler lösen kann, kann die Kalibrierung abgeschlossen werden.

\newpage

Die Synapsen eines neuronalen Netzes, also die Verbindungen zwischen den Neuronen, sind nicht statisch und können sich während des Lernprozesses verändern. Es können neue Synapsen entstehen, bestehende Kontakte können ihren Übertragungswiderstand ändern oder ganz aufgelöst werden. Besonders bei der handschriftlichen Erkennung haben neuronale Netze in den letzten Jahren bessere Ergebnisse geliefert als die konkurrierenden Verfahren. Durch eine zusätzliche Erhöhung der Rechenleistung mittels GPU-Implementierungen, konnten Wissenschaftler mit dieser Methode viele Wettbewerbe der Mustererkennung gewinnen. \cite{OCRB}

\subsubsection{ICR und IWR}
Bei der Intelligent Character Recognition werden die digitalen Abbilder der Zeichen mit Wörterbüchern verglichen und anschließend wird die wahrscheinliche Fehlerfreiheit der Zeichen analysiert. Abhängig von dieser wahrscheinlichen Fehlerfreiheit wird das Zeichen identifiziert oder einer erneuten Bewertung unterzogen.

Die Intelligent Word Recognition vergleicht Einzelzeichen, die nicht separat erkannt werden können, mit Wörterbüchern. Die Erkennungsgenauigkeit ist von der Größe des eingebundenen Wörterbuchs abhängig. IWR liefert besonders bei der Handschrifterkennung gute Erfolge.

\subsubsection{Implementierung}
In den folgenden Kapiteln ist die Implementierung des OCR-Moduls beschrieben.
\paragraph{Tesseract OCR und pytesseract-Modul}
Der erste Schritt zur Implementierung war die Evaluierung der geeigneten OCR-Software. Diese wurde im Evaluierungssprint des Projektes evaluiert. Es wurden drei Softwarepakete evaluiert: Tesseract OCR (pytesseract), OCRopus und OmniPage Ultimate. Nach dem Vergleich dieser Lösungen hinsichtlich mehrerer Kriterien fiel die Entscheidung auf die tesseract-Engine und das dazugehörige Python-Modul pytesseract.

Um Tesseract-OCR (engine) zu installieren, installiert man die folgenden Pakete auf dem Server:
\begin{itemize}
\item tesseract-ocr
\item tesseract-ocr-deu
\end{itemize}

\newpage

Zum Testen der Analyse führt man tesseract mit folgendem Befehl in der Commandline aus:
\begin{lstlisting}[caption={tesseract-Ausführung}, language=bash]
tesseract test3.jpg out3
\end{lstlisting}
\cite{TESS1} \cite{TESS2}

Hier erkennt man das analysierte Bild, sowie den Output der tesseract-Engine.

\insertpicture{images/ocr/fresh.jpg}{Beispielbild für OCR-Analyse}{(selfmade)}{itm:fresh}{0.8}

Output:

THE\_FRESH\_STEP

Adamski, Coric, Schwertberger

Entwicklung einer neuen Webseite.

die alle Abteilungen. Werlstatten und A / m

Für die Installation der Python-Anbindung pytesseract sind folgende Befehle notwendig.

\begin{lstlisting}[caption={pytesseract-Installation}, language=bash]
apt-get install python3
apt-get install python3-pip
pip3 install pytesseract
apt-get install python3-pil
\end{lstlisting}
\cite{PYTES} \cite{PYTES2} \cite{PYTES3}

Mit einem Testscript in Python kann die optische Zeichenanalyse ausgeführt werden.

\begin{lstlisting}[caption={pytesseract-Code}, language=Python]
import pytesseract
from PIL import Image
print (pytesseract.image_to_string(Image.open('test3.jpg')))
\end{lstlisting}

\paragraph{Implementierung des OCR-Moduls}
Um Bilder hochladen zu können und eine Bild-zu-Text-Analyse zu ermöglichen, wurde ein OCR-Modul implementiert. Die Idee dahinter ist, dass die Benutzer unserer Hefte möglichst einfach Bilder auswählen und einer Zeichenanalyse (OCR) unterziehen können. Dabei werden prinzipiell die Dateiformate .jpeg .png und .gif akzeptiert.

Bei der optischen Zeichenanalyse (OCR) wird der am Bild vorhandene Text durch eine OCR-Engine analysiert und in ein Textformat umgewandelt. Für unsere Lösung wurde das python-Framework pytesseract mit der OCR-Engine tesseract benutzt.

Um ein Bild zu analysieren, klickt der User zunächst auf folgenden Button im Heft:

\insertpicture{images/ocr/bu.png}{OCR-Button}{(selfmade)}{itm:but}{0.6}

Darauf öffnet sich ein Dialog mit einem Dateieingabefeld:

\insertpicture{images/ocr/di.png}{OCR-Dialog}{(selfmade)}{itm:di}{0.8}

Nachdem auf den Button ,,Analysieren" geklickt wurde, wird das Bild auf den Server geladen und mittels OCR analysiert. Das Ergebnis dabei ist ein neu erstelltes Textelement im Heft mit dem textuellen Inhalt des angegebenen Bildes.

Wie schon erwähnt, erscheint bei Klick auf den OCR-Button ein Dialog, in welchem die zu analysierende Bilddatei angeben werden kann. Bei Klick auf den Button ,,Analysieren" wird folgende Subroutine aufgerufen:

\begin{lstlisting}[caption={Upload OCR-File}, language=Python]
$scope.uploadOCRFile = function(){
        var file = $scope.ocrFile;
        if((file.type == "image/jpeg" || file.type == "image/png" ||
        	file.type == "image/gif") && file.size < 5242880) {
            $scope.errormessage = "";
            var uploadUrl = "/api/analyseOCR";
            var message = fileUpload.uploadFileToUrl(file, uploadUrl);
            message.then(function(data) {
                data_data = "{\"data\":\""+data['ocrt']+"\"}";
                $scope.addelement('textarea', data_data);
                $window.location.reload();
                $window.location.reload();
            });
            ngDialog.close({
                template: 'ocrFileDialog',
                controller: 'notebookEditCtrl',
                className: 'ngdialog-theme-default',
                scope: $scope
            });
        }else{
            if(file.size > 5242880) {
                $scope.errormessage = "file size is more than 5MB";
            }else{
                $scope.errormessage = "filetyp is not supported";
            }
        }
    };
\end{lstlisting}

Hier ist zu sehen, dass zunächst der Dateityp und die Dateigröße überprüft wird. Wenn alles den Kriterien entspricht wird die Methode \textit{uploadFileToUrl()} aufgerufen. Unter der URL \textit{url(r'\^{}api/analyseOCR', 'dsn.views.notebook\_views.view\_analyseOCR', name=,,analyseOCR")} findet sich die Python-Methode \textit{view\_analyseOCR()}.

\newpage

In dieser Methode erhält das File einen neuen, zufällig generierten Namen und wird auf den Server lokal hochgeladen. Dann wird schließlich die Methode, die für das eigentliche Analysieren zuständig ist, aufgerufen:

\begin{lstlisting}[caption={OCR-Analyse}, language=Python]
ocrtext=analyseOCR(os.getcwd()+"/dsn/static/upload/"+filename+"."+typ)
\end{lstlisting}

Schließlich kommt in der Methode analyseOCR das pytesseract-Framework zum Einsatz:

\begin{lstlisting}[caption={analyseOCR mittels pytesseract}, language=Python]
def analyseOCR(file):
	s = str(pytesseract.image_to_string(
		Image.open(file)).encode(sys.stdout.encoding,
		errors='replace'))
    s = s[2:-1]
    s = s.replace("\\n", "<br />").replace("\\x", "")
    return s
\end{lstlisting}

Nach der Analyse wird der erkannte Text als JSONResponse zurückgegeben und das lokal hochgeladene Bild wird wieder vom Server gelöscht.

\begin{lstlisting}[caption={Fileentfernung und Rückgabe des OCR-Textes}, language=Python]
os.remove(os.getcwd()+"/dsn/static/upload/"+filename+"."+typ)
return JsonResponse({'ocrt': ocrtext})
\end{lstlisting}

Abschließend wird mit dem daraus analysierten Text ein neues Textelement im Heft erzeugt. Dazu wird die \textit{addelement()} Methode aufgerufen. Aus folgendem Bild ergibt sich dann das folgende Textelement.

\insertpicture{images/ocr/bspbild.jpg}{Beispielbild für OCR-Analyse}{(selfmade)}{itm:bsp}{0.55}

\insertpicture{images/ocr/bsptext.png}{Erzeugtes Textelement}{(selfmade)}{itm:text}{0.35}

\subsubsection{Ausblick}
Eine Optimierung des OCR-Moduls hinsichtlich Erkennungsgenauigkeit und Handschrifterkennung wurde angedacht. In Zukunft könnte man das OCR-Modul durch entsprechendes (automatisiertes) Training in dieser Hinsicht weiterentwickeln. Eine höhere Fehlertoleranz bei Bildern mit schlechter Qualität ist ebenso wünschenswert.



\newpage %----------------------------------------------------------------------------------------------

\section{Auswertung und Benchmarks}
\label{sec:auswertung}
%\section*{Auswertung und Benchmarks}
\cfoot{Niklas Hohenwarter}
Dieses Kapitel befasst sich mit den messbaren Ergebnissen des Projektes und der technischen Auswertung. Es wird überprüft, ob die richtigen Frameworks zur Realisierung verwendet wurden und ob die Software ausreichend performant ist und vom Benutzer akzeptiert wird.

\subsection{Benutzerakzeptanz}
Um die Benutzerakzeptanz bzw. das Interface Design zu überprüfen, existieren eine Vielzahl von Richtlinien. Des Weiteren gibt es auch ganze Bücher und Papers, welche sich mit diesen Themen befassen. In unserem Projekt haben wir uns nur auf zehn Richtlinien\cite{INTDES1} beschränkt, an welche wir uns halten wollten. 

Es ist wichtig, dem User mittels einem eindeutigen Design die Website zu vermitteln. Dieses Design soll vom Benutzer wiedererkannt werden, damit er immer weiß, ob er sich gerade auf der DSN Website befindet. Dazu wurde innerhalb der ganzen Applikation grün in großen Mengen als Signalfarbe verwendet. Des Weiteren ist in jeder Ansicht außer der Heftansicht immer das DSN Logo sichtbar um den Wiedererkennungswert zu steigern.

Eine eindeutige Benutzerführung ist ebenfalls essentiell. Aus diesem Grund ist das Registrierungsformular direkt auf der Startseite positioniert. Nach der Registrierung erhält der Benutzer eine Email, welche ihn zum Login leitet. Von dort aus wird der Benutzer zum Stundenplan geleitet, welchen er dann eintragen kann. Über ein großes Hauptmenü am oberen Bildschirmrand können alle anderen gewünschten Funktionen erreicht werden.

Eine klare Call-to-Action Führung findet sich ebenfalls auf der Website. Ein in Zukunft erstelltes Werbevideo klärt dem Benutzer beim ersten Besuch der Website über die Funktionen von DSN auf. Im Video wird der User zu einer Registrierung animiert.

Das Feedback an den User fehlt leider ein bisschen auf der DSN Website. Es ist nicht immer klar, ob die Website gerade lädt oder ob einfach nichts passiert. Hier werden in Zukunft noch Ladeanimationen eingefügt, um den Benutzer über einen Ladevorgang zu informieren.

Es ist wichtig, bei der Registrierung so wenig Informationen wie möglich abzufragen, um die Hemmschwelle der Registrierung möglichst gering zu halten. Dies ist vor allem durch die OAuth Registrierung gut realisiert, da hier nur ein Button gedrückt werden muss.

Die Applikation hat in allen Eingabefeldern Default-Werte, um den User besser durch das Formular zu führen. Dies hilft dem User beim Ausfüllen und erklärt ihm, welche Daten eingegeben werden müssen.

Der User sollte immer das Gefühl haben zu wissen, was mit seinen eingegebenen Daten passiert. Dies ist notwendig, um die Eingabe von richtigen Daten zu fördern und um dem User zu erklären, wozu die Daten benötigt werden.

Um den Ärger über die Applikation gering zu halten gibt es keinen Button, welcher das gesamte Formular zurücksetzt. Somit kann der User nicht aus Versehen alle mühsam eingegebenen Formularfelder unabsichtlich löschen.

In der DSN Applikation gibt es viele eindeutige Fehlermeldungen, um den Benutzer über seine Fehleingabe aufzuklären und um eine richtige Eingabe zu fördern.

In DSN gibt es beabsichtigt keine Breadcrumbs, da diese den Benutzer oft verwirren. Die Führung der Menüs ist einfach, somit werden diese auch nicht benötigt.

\subsection{Änderungsvorschläge}
Im Nachhinein betrachtet wäre es eine gute Idee gewesen, das Full Stack Framework MeteorJS zu verwenden. Durch die Verwendung von Meteor hätten sich im Laufe des Projektes einige Vorteile ergeben. 

Zum Einen war das Team bezüglich der vielschichtigen Softwarearchitektur verwirrt. Es war für viele nicht klar, welche Änderung sich auf welchen Teil der Software auswirkt und wie diese Änderung aktiv geschalten wird. Dadurch wurde viel Zeit verloren. Hier hätte ein Full Stack Framework den Vorteil gehabt, dass die gesamte Software mit einem Befehl zentral deploybar gewesen wäre. 

Des Weiteren bietet Meteor viele Features, die händisch implementiert wurden. So ist die Benutzerverwaltung oder die Anmeldung über OAuth bereits ein fertiges Modul in Meteor. Außerdem werden die Daten automatisch zwischen Server und Client synchron gehalten, wodurch es um einiges einfacher gewesen wäre, das PWS zu integrieren.

\newpage %----------------------------------------------------------------------------------------------

\section{Ausblick}
\label{sec:ausblick}
%\section*{Ausblick}
\cfoot{Thomas Stedronsky}
Um den stetigen Verbesserungsprozess von DNS zu gewährleisten, gibt es weitere Ideen, um die Software mit mehr Funktionalität auszustatten und diese somit noch attraktiver für die Nutzer zu machen. 
\subsection{Bezahlsystem}
Um das System in die Wirtschaft zu bringen, müsste ein Bezahlsystem eingebaut werden, das die Kostenabwicklung übernimmt. Dieses System wäre die Vorrausetzung, um zahlende Kunden für das Produkt zu gewinnen.\\
Es gibt bereits ein Geschäftmodell und anhand dessen dieses System aufgebaut sein müsste. 
\insertpicture{images/ausblick/geschaeftsmodell}{Geschäftsmodell}{(selfmade)}{itm:geschaeftsmodell}{1.0}
Die Bezahlung sollte über bekannte Online-Bezahllösungen wie PayPal, Amazon Payments, etc. abgewickelt werden, um sich rechtlich abzusichern. Durch dieses Bezahlsystem wäre es dann möglich, die 90 Tage Testaccounts auf Pro Accounts upzugraden. Außerdem gibt es extra für Schulen zugeschnittene Lizenzen, mit denen Schulen eine zeitlich unbegrenzte Lizenz erwerben können, wobei die Schule dann das Hosting des Servers selber übernehmen muss. 
\subsection{Einblendung von Werbung}
Durch die Einblendung von Werbung in den 90 Tage Testversionen sollen neue Einnahmequellen geschaffen werden. Dadurch sollten diese Probeaccounts finanziert werden. Durch diese Verbesserung wäre es vorstellbar, dass User, die störungsfrei arbeiten wollen, ihren 90 Tage Probeaccount schneller verlängern und somit dauerhafter Nutzer von DigitalSchoolNotes werden. Diese Werbungen sollen über das Google AdSense Werbenetzwerk bezogen werden. Dies ist eine gängige Methode in Webapplikation, um Werbung einzubinden.
\subsection{Zeichenelement}
Um die Palette an Elementen weiter aufzustocken, gibt es die Idee eines Zeichenelements. Mit diesem Element soll es dem User möglich sein, direkt im Heft ein Skizze anzulegen und diese dann nach Belieben in späterer Folge weiter zu editieren. Dadurch würde Einiges an Aufwand wegfallen, um eine Skizze, beziehungweise eine Zeichnung, in einem Heft zu platzieren.\\
Hierbei soll gefördert werden, dass die Nutzer Gedanken im Sinne von Skizzen, Mindmaps, etc. schnell in einem Heft anlegen können, wie sie es auch in einem handgeschriebenen Heft machen würden. Mit diesem Tool würde man näher an die Idee einer handschriftlichen Mitschrift heranrücken.
\subsection{Videoeinbindung}
Um das Heft noch interaktiver zu gestalten könnte man die Elemente um ein Video-Element erweitern. Mit diesem Element wäre es dem User möglich, zum Beispiel Videos, die den Unterricht betreffen, in das interaktive Schulheft einzubinden. Somit könnten Video Tutorials direkt im Schulheft angesehen werden. Um Ressourcen zu sparen, wäre die Idee, diese Videos nicht auf dem DigitalSchoolNotes Server abzuspeichern, sondern lediglich Links in das Heft einzubinden. Somit müssten nur Links aus Videoportalen eingebunden werden und die dann anschließend direkt im Heft abgespielt werden. Durch diese Erweiterung der Elemente könnte man klare Abgrenzungen zu normalen Textdokumenten schaffen, wodurch ein interaktives Notizheft mit vielen verschiedenen Inhalten entstehen würde.
\subsection{Drag and Drop von Bildern}
Durch eine zusätzliche Drag and Drop Funktion um Bilder einzufügen, könnte dem User viel Aufwand abgenommen werden. Somit müsste ein Bild nur mehr in das Heft hineingezogen werden und das Bild wäre im Heft platziert. 
\subsection{PDF Download}
Durch die Möglichkeit eines PDF-Downloads des Heftes wäre es dem User möglich geschriebenen Inhalt auch offline verfügbar zu haben. Durch diese Methode könnte man das jeweilige Heft zum Beispiel ausdrucken und archivieren. Außerdem würde mit dieser Funktion das Problem der maximalen Heftanzahl aufgelockert werden. Es wäre anschließend möglich, nicht mehr gebrauchte Hefte als PDF herunterzuladen und danach zu löschen. Somit könnte Platz für aktuelle Fächer beziehungsweise Hefte geschaffen werden.
\subsection{Shortcuts}
Für eine bessere Bedienbarkeit von DigitalSchoolNotes könnten Shortcuts in das System integriert werden, um so schneller gewünschte Funktionen auszuführen. Somit könnten beispielsweise einfach Elemente mittels einer simplen Tastenkombination in das Heft eingebunden werden. Durch diese geringfügige Verbesserung würde der Userkomfort gesteigert werden. Durch dieses Feature könnte effizienter mit der Web-Applikation DigitalSchoolNotes gearbeitet werden.\\
Es könnte auch unter anderem eine Suchfunktion innerhalb eines Heftes geben, wie es in den meisten Systemen mit Strg + F üblich ist. Dadurch könnten einzelne Elemente untersucht werden und dann automatisch zu diesen verwiesen werden. Damit wäre es möglich, besser und schneller in z.B. geteilten Heften nach dem passenden Inhalt zu suchen. 
\subsection{Filehoster-Anbindung}
Derzeit ist es möglich zum Beispiel Bilder in ein Heft hochzuladen. Diese Bilder werden dann in der Amanzon S3 Cloud(siehe \ref{sec:bildelement}) abgespeichert. Allerdings benötigen diese Bilder trotz Komprimierung eine Menge an Ressourcen. Um dem User zu ermöglichen, beliebig viele Bilder in sein Heft zu importieren, gibt es eine Möglichkeit. Durch die Anbindung an ein Filehosting-System, wie beispielsweise Dropbox, könnten Bilder direkt aus dem Shared Folder in das Heft eingebunden werden. Dadurch könnte jeder User seinen Speicher von außerhalb verwenden und hätte somit alle Dateien auf dem selben Ort abgelegt.\\
Dies hätte den großen Vorteil, dass schnell und effizient mehrere Systeme gleichzeitig auf diese Medien zugreifen können. Durch die Verbindung mit bereits verwendeter Software könnte der Benutzer einiges an Administrationsaufwand abgenommen werden. Dadurch könnte man DigitalSchoolNotes überall verwenden und hätte alle Dateien, die sich in solchen Shared Foldern befinden, jederzeit zur Verfügung. 

\newpage

\subsection{Web Untis Anbindung}
Da DigitalSchoolNotes als Zielgruppe Schulen hat, sollte darüber nachgedacht werden, eventuell Stundenplansystem wie WebUntis in die Web-Applikation einzubinden. Somit müsste der Stundenplan auf der DSN Website nicht mehr manuell eingegeben werden, sondern würde direkt nach Verbindung mit dem Web Untis System der Schule synchronisiert werden.\\
Durch diese Neuerung würde dem User einiges an Aufwand abgenommen werden. Es müssten maximal kleine Anpassungen vorgenommen werden, um den Stundenplan im selben Rahmen zu nutzen, wie ihn Web Untis zur Verfügung stellt. Dadurch müssten lediglich die Hefte mit dem neuen Stundenplan verbunden werden und danach könnte man diesen genau wie einen manuell eingegeben Stundenplan nutzen. 

\subsection{LDAP-Anbindung}
Um bereits bestehende Lösungen, die in einer Schule eingesetzt werden, zu nutzen, gibt es die Idee, bestehende LDAP Systeme einer Schule mit DigitalSchoolNotes zu verbinden. Somit müssten sich die Nutzer nicht mehr extra für DSN anmelden, sondern könnten bereits verwendete Accounts verwenden. Dadurch verringert sich der Administrationsaufwand drastisch.

\subsection{Optimierung des PWS}
Um das Parallel Working System weiter zu verbessern, gibt es Ideen, um dieses System schneller und performanter zu machen.

Um eine höhere Performance zu erlangen, gibt es die Möglichkeit, das Aktualisierungsverfahren zu verändern. Die Aktualisierung in gleichbleibenden Abständen ist nicht die performanteste Lösung. Es gibt ein Pull und Push Prinzip, wie es beispielsweise die Google Realtime API verwendet. Mit diesem Verfahren wird nach jeder Änderung am System eine Push-Benachrichtung an den Server geschickt. Anschließend werden die aktiven Nutzer benachrichtigt und der geänderte Inhalt kann aktualisiert werden.

Um dieses Verfahren in das Parallel Working System zu integrieren, müsste das System an mehreren Stellen angepasst werden, wodurch das System aber effizienter arbeiten würde und somit noch näher an eine Echtzeit-Kommunikation herankommen würde.\\
Diese Änderung hätte zur Folge das sich keine Differenz zwischen Änderung und Aktualisierung mehr ergäben, wodurch besser gemeinsam an einer Heftseite gearbeitet werden könnte.
\newpage
Um weiteren Nutzerkomfort zu schaffen, müsste das Prinzip, dass nur ein Nutzer pro Element arbeiten darf, optimiert werden. Es gäbe die Möglichkeit, diese Regel aufzuheben und ein gleichzeitiges Arbeiten an einem Element zu erlauben. Dies wäre der letzte Schritt zur endgültigen Echtzeit-Kommunikation. Allerdings müsste mit dieser Änderung das ganze Konzept von DigitalSchoolNotes überarbeitet werden, um dies zu ermöglichen. Denn mit der derzeitigen Version wird erst ab Verlassen des Bearbeitungsmodus der aktuelle Stand gespeichert und für alle Nutzer im Heft zur Verfügung gestellt. Mit diesem neuen Prinzip müsste jeder Tastendruck in Echtzeit an die aktiven Nutzer übermittelt werden, um keine Konflikte aufkommen zu lassen. Diese Funktion könnte nur mit einem dementsprechenden Framework durchgeführt werden. Es müsste eine Machbarkeitsstudie durchgeführt werden, um abzuklären, ob eine solch aufwendige Änderung mit dem aktuellen System umsetzbar wäre.

\subsection{OCR-Modul Optimierung}
Eine Optimierung des OCR-Moduls hinsichtlich Erkennungsgenauigkeit und Handschrifterkennung wurde angedacht. In Zukunft könnte man das OCR-Modul durch entsprechendes (automatisiertes) Training in dieser Hinsicht weiterentwickeln. Eine höhere Fehlertoleranz bei Bildern mit schlechter Qualität ist ebenso wünschenswert.

\newpage %----------------------------------------------------------------------------------------------

\section{Zusammenfassung}
\label{sec:zusammenfassung}
%\section*{Zusammenfassung}
\cfoot{Thomas Stedronsky}
DigitalSchoolNotes ist eine Web-Applikationen zur einfachen Führung einer digitalen Mitschrift.\\\\
Für die Registrierung bei DSN gibt es 2 Möglichkeiten. Einerseits gibt es die Möglichkeit sich mittels Eingabe der E-Mail Adresse sowie Name und Passwort zu registrieren. Die weitaus einfachere Variante ist hierbei allerdings die Registrierung mit OAuth. Dabei verwendet man ein bestehendes Konto eines sozialen Netzwerkes wie Facebook oder Google+, um sich bei DSN zu registrieren. Anschließend kann man sich erfolgreich in die Applikation einloggen.\\
\\
Ist man zusätzlich zu einem normalen User noch ein Administrator gibts es die Möglichkeit der Nutzung einer eigenen Plattform, um die Nutzer zu administrieren. Mit diesem Interface ist es dem Administrator möglich User zu benachrichtigen oder diese bei Bedarf zu löschen. Außerdem können die Rechte der einzelnen Nutzer manuell administriert werden.\\
\\
Für eine optimale schulische Organisation ist ein Stundenplan implementiert. In diesen können manuell Unterrichtszeiten sowie Fächer, Räume und Lehrer eingetragen werden. Diese dynamik wurde gewählt, weil diese Informationen Schulspezifisch sind. Dadurch geben wir den Nutzer mehr Freiraum in der Gestalung seines eigenen Stundenplans.\\
\\
Die weitaus wichtigste Funktion ist das Hefte anlegen und diese dann zu bearbeiten. Es können Hefte angelegt werden die einen bestimmten Namen erhalten und es wird entschieden welche Sichtbarkeit das Heft hat entweder öffentlich oder privat. Diese Angabe ist wichtig, denn öffentliche werden auf dem DSN-Profil angezeigt.\\
\\
In dem erstellten Heft können dann diverse Elemente eingebunden werden, um den Inhalt in das Heft zu bringen. Diese verschiedenen Elemente können beliebig im Heft platziert werden. Durch klicken und ziehen des Elements kann das jeweilige Element verschoben werden. Mittels Hover können Elemente schnell und leicht wieder aus dem Heft entfernt werden.\\
\\
Um einfachen Text in das Heft einzufügen gibt es das Textelement. Mit diesem ist es dem Benutzer möglich einfachen Text einzutragen und diesen dann entsprechende zu editieren. Um den Text au0erdem hervorzuheben können Effekte wie Fett, Kursiv, Unterstrichen, etc. auf den Text angewendet werden. Des weiteren können Tabellen in dieses Element eingefügt werden. Durch die hereinnahme von Hyperlinks kann zusätzlicher Inhalt außerhalb von DigitalSchoolNotes angemerkt werden.\\
\\\\\\
Da in einem Heft Erklärungsgrafiken oder zusätzliche Medien nicht fehlen dürfen gibt es ein eigenes Bild-Element. Bilder können auf den DigitalSchoolNotes Server hochgeladen werden und anschließend in das Heft eingebunden werden. Die Bilder können eine beliebige Größe annehmen und beliebig im Heft platziert werden. Die Größe kann laufend geändert werden.\\
\\
Dadurch, dass eine digitale Mitschrift besonders in technischen Schulen vorkommt gibt es ein eigenes Codeelement mit diesen Code Snippets eingebunden werden können. Diese Codezeilen können je nach Programmiersprache oder Scriptsprachen verschieden hervorgehoben werden. Das Codeelement unterstützt unter anderem XML, HTML, JAVA, JavaScript, PHP, Python, C, C++, C\# und SQL. Um noch eine bessere Lesefreundlichkeit zu schaffen sind auf der linken Seite Zeilennummern automatisch integriert, um den Code besser im Überblick zu behalten und eventuelle Referenzierungen anlegen zu können. Dadurch können einzelne Codeabschnitte optimal in die digitale Mitschrift integriert werden\\
\\
Es kann vorkommen, dass man einen gewissen Teil aus öffentlichen Heften interresant findet und diesen gerne im eigenen Heft hätte, kein Problem. Mithilfe eines Features ist es möglich Seiten aus öffentlichen Heften in mein eigenes Heft zu importieren. Dadurch können Inhalte geteilt werden und anschließend erweitert werden. Des weiteren können Inhalte von eigenen Heften in alle Hefte die ich besitze importiert werden.\\
\\
Um bereits niedergeschriebenen Text weiter zu editieren gibt es eine Funktion, welche die optische Zeichenerkennung(Optical Character Recognitions) unterstützt. Mithilfe dieses Features kann der User gedruckten Text in bearbeitbaren Text umwandeln. Durch eine OCR-Engine kann ein Bild auf den Server hochgeladen werden, dass nach der Analyse wieder gelöscht wird. Das hochgeladene Bild wird von der integrierten OCR-Engine analysiert und anschließend in ein Textelement ausgegeben. Dieses Element kann nach der Analyse wie ein normales Textelement verwendet werden. Somit kann an bereits bestehenden Text mühelos weitergearbeitet werden\\
\\
Gruppenarbeiten oder Referate mit Mehrpersonenteams sind im Schulaltag kaum mehr wegzudenken. Darum gibt es mit DigitalSchoolNotes das Parallel Working System. Mit diesem System ist es möglich Hefte mit anderen User zu teilen und diese dann gleichzeitig zu bearbeiten. Mit diesem System kann ich beliebig viele Nutzer an der Bearbeitung meines Heftes teilhaben lassen. Diese User haben dann die Möglichkeit geschrieben Heftinhalt anderer User in Echtzeit zu verfolgen. Durch dieses integrierte System steht Gruppenarbeiten nichts mehr im Weg.\\
\\
Unleash your Productivity!
%Text...

\newpage %----------------------------------------------------------------------------------------------

\cfoot{}
\pagenumbering{roman}
\section*{Anhang}
\label{sec:anhang}

\subsection*{Glossar}
\label{subsec:glossar}
\begingroup
\renewcommand{\section}[2]{}
\printglossary[style=tree]
\endgroup
\newpage

{\small\color{white}.}
\vspace{-2cm}
\subsection*{Abbildungen}
\label{subsec:abbildungen}
\begingroup
\renewcommand{\section}[2]{}
\listoffigures
\endgroup

\subsection*{Listings}
\label{subsec:listings}
\begingroup
\renewcommand{\section}[2]{}
\lstlistoflistings
\endgroup

\subsection*{Quellen}
\label{subsec:quellen}
\begingroup
\renewcommand{\section}[2]{}
\bibliographystyle{alpha}
\bibliography{sources}
\endgroup

\end{document}

%\Pictures
%\insertpicture{path}{Name=Caption}{Quelle}{Ref}{Skalierung}
%\insertpicture{images/exampleimg}{Example Image}{(selfmade)}{itm:pic1}{1.0}

%\References (Sections, Listings)
%\ref{sec:example}
%\ref{HelloWorld}

%\Code-Listing
%\begin{lstlisting}[caption = Description, label = Label]
%\begin{lstlisting}[caption = A simple hello world example, label = HelloWorld]

%\Quellen
%\cite{book123}
%\cite{cite_key}
