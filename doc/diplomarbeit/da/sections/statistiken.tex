\cfoot{Thomas Stedronsky}

\subsection{Statistiken}
Im Zuge der Marktforschung wurden Schüler aus den dritten, vierten und fünften Jahrgängen der Abteilung Informationstechnolgie im TGM befragt. Diese Befragung bezog sich auf das Mitschreiben im Unterricht. 

Dabei wurden folgende Fragen gestellt:

1) Ich schreibe regelmäßig im Unterricht mit\\
	\begin{tabular}[t]{lr}
	O Ja, am Papier\\
	O Ja, am Laptop\\
	O Nein
	\end{tabular}
	

2) Meine Mitschrift ist gut strukturiert\\
	\begin{tabular}[t]{lr}
	O Ja\\
	O Nein
	\end{tabular}

3) Ich verwende das Produkt Microsoft OneNote zur Mitschrift am Laptop\\
	\begin{tabular}[t]{lr}
	O Ja\\
	O Nein
	\end{tabular}
	
4) Ich verwende das Produkt Microsoft Word um mitzuschreiben und kann damit meine Mitschrift gut strukuturieren\\
	\begin{tabular}[t]{lr}
	O Ja\\
	O Nein
	\end{tabular}
	
Um eine aussagekräftiges Ergebnis zu erreich wurden 163 Schüler im Zeitraum vom 13. April bis zum 14. April befragt.
\newpage

Bei dieser Befragung kam es zu folgenden Ergebnis:
\insertpicture{images/statistik/statistik1.png}{Statistik 1 und 2}{(selfmade)}{itm:menu_main}{1.0}
In der Statistik 1 ist ersichtlich, dass 76.7\% digital oder gar nicht mitschreiben, das ist der Markt für DigitalSchoolNotes.
Diese Schüler sind die Zielgruppe von DSN, wird bereits mitgeschrieben so soll der User davon überzeugt werden, dass DSN die bestmögliche Lösung zur Führung einer digitalen Mitschrift ist. Außerdem sollen Schüler bewegt werden statt gar nicht mit DSN mitzuschreiben.

Bei der zweiten Statistik wird gezeigt wie zufrieden die Schüler mit ihrer Mitschrift sind. 42.9\% der Schüler haben eine nicht gut strukturierte Mitschrift. DSN bietet die Möglichkeit eine gut strukturierte Mitschrift zu führen. 
\insertpicture{images/statistik/statistik2.png}{Statistik 3 und 4}{(selfmade)}{itm:menu_main}{1.0}
Bei den Statistiken 3 und 4 wurde nach Konkurenzprodukten beziehungsweise deren Nutzung gefragt. Dabei kam heraus, dass nur eine sehr kleine Anzahl von 3.6\% Microsoft OneNote zum Mitschreiben nutzt. Außerdem sind drei Viertel der Befragten mit ihrer Microsoft Word Mitschrift zufrieden und betrachten diese als sinnvoll. 
\newpage
