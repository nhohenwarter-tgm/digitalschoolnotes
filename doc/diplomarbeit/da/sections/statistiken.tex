\cfoot{Thomas Stedronsky}

\subsection{Statistiken}
Im Zuge der Marktforschung wurden Schüler aus den dritten, vierten und fünften Jahrgängen der Abteilung Informationstechnolgie am TGM befragt. Die ersten und zweiten Jahrgänge wurden nicht befragt, weil diese keine Laptops im Unterricht nutzen. Diese Befragung bezog sich auf das Mitschreiben im Unterricht. 

Es wurden folgende Fragen gestellt:

1) Ich schreibe regelmäßig im Unterricht mit\\
	\begin{tabular}[t]{lr}
	O Ja, am Papier\\
	O Ja, am Laptop\\
	O Nein
	\end{tabular}
	

2) Meine Mitschrift ist gut strukturiert\\
	\begin{tabular}[t]{lr}
	O Ja\\
	O Nein
	\end{tabular}

3) Ich verwende das Produkt Microsoft OneNote zur Mitschrift am Laptop\\
	\begin{tabular}[t]{lr}
	O Ja\\
	O Nein
	\end{tabular}
	
4) Ich verwende das Produkt Microsoft Word um mitzuschreiben und kann damit meine Mitschrift gut strukuturieren\\
	\begin{tabular}[t]{lr}
	O Ja\\
	O Nein
	\end{tabular}
	
Um ein aussagekräftiges Ergebnis zu erreichen, wurden 163 Schüler im Zeitraum vom 13. April bis zum 14. April befragt.
\newpage

Bei dieser Befragung kam es zu folgendem Ergebnis:
\insertpicture{images/statistik/statistik1.png}{Fragestellung 1 und 2}{(selfmade)}{itm:menu_main}{1.0}
In der Fragestellung 1 ist ersichtlich, dass 76.7\% digital oder gar nicht mitschreiben
Diese Schüler sind die Zielgruppe von DSN. Wird bereits digital mitgeschrieben, so soll der Nutzer davon überzeugt werden, dass DSN die bestmögliche Lösung zur Führung einer digitalen Mitschrift ist. 

Bei der zweiten Fragestellung wird gezeigt, wie zufrieden die Schüler mit ihrer Mitschrift sind. 42.9\% der Schüler haben eine nicht gut strukturierte Mitschrift. DSN bietet die Möglichkeit eine gut strukturierte Mitschrift zu führen. 
\insertpicture{images/statistik/statistik2.png}{Fragestellung 3 und 4}{(selfmade)}{itm:menu_main}{1.0}
Bei den Fragestellungen 3 und 4 wurde nach Konkurenzprodukten beziehungsweise deren Nutzung gefragt. Dabei kam heraus, dass nur eine sehr kleine Anzahl von 3.6\% Microsoft OneNote zum Mitschreiben nutzt. Außerdem sind drei Viertel der Befragten mit ihrer Microsoft Word Mitschrift zufrieden und betrachten diese als sinnvoll. 
\newpage
