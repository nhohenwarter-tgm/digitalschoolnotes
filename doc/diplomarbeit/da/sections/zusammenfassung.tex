%\section*{Zusammenfassung},
\cfoot{Thomas Stedronsky}
DigitalSchoolNotes ist eine Web-Applikationen zur einfachen Führung einer digitalen Mitschrift.

Für die Registrierung bei DSN gibt es zwei Möglichkeiten. Einerseits gibt es die Möglichkeit, sich mittels Eingabe der E-Mail Adresse, sowie Name und Passwort zu registrieren. Die weitaus einfachere Variante ist hierbei allerdings die Registrierung mit OAuth. Dabei verwendet man ein bestehendes Konto eines sozialen Netzwerkes wie Facebook oder Google+, um sich bei DSN zu registrieren. Anschließend kann man sich erfolgreich in die Applikation einloggen.

Ist man zusätzlich zu einem normalen User noch ein Administrator, gibt es die Möglichkeit der Nutzung einer eigenen Plattform, um die Nutzer zu administrieren. Mit diesem Interface ist es dem Administrator möglich, User zu benachrichtigen oder diese bei Bedarf zu löschen. Außerdem können die Rechte der einzelnen Nutzer manuell administriert werden.

Für eine optimale schulische Organisation ist ein Stundenplan implementiert. In diesen können manuell Unterrichtszeiten, sowie Fächer, Räume und Lehrer eingetragen werden. Dadurch geben wir dem Nutzer mehr Freiraum in der Gestaltung seines eigenen Stundenplans.

Die weitaus wichtigste Funktion ist das Anlegen von Heften und die Bearbeitung dieser. Es können Hefte angelegt werden, die einen bestimmten Namen erhalten und es wird entschieden, welche Sichtbarkeit das Heft hat (öffentlich oder privat). Auf dem DSN-Profil werden öffentliche Hefte angezeigt.

In dem erstellten Heft können dann diverse Elemente eingebunden werden. Diese verschiedenen Elemente können beliebig im Heft platziert werden. Durch Klicken und Ziehen des Elements kann das jeweilige Element verschoben werden. Mittels Hover können Elemente schnell und leicht wieder aus dem Heft entfernt werden.

Um einfachen Text in das Heft einzufügen, gibt es das Textelement. Mit diesem ist es dem Benutzer möglich, einfachen Text einzutragen und diesen dann entsprechend zu editieren. Um den Text außerdem hervorzuheben, können Effekte, wie fett, kursiv, unterstrichen, etc. auf den Text angewendet werden. Des Weiteren können Tabellen in dieses Element eingefügt werden. Durch Hyperlinks kann zusätzlicher Inhalt außerhalb von DigitalSchoolNotes angemerkt werden.\\
Da in einem Heft Erklärungsgrafiken oder zusätzliche Medien nicht fehlen dürfen, gibt es ein eigenes Bild-Element. Bilder können auf den DigitalSchoolNotes Server hochgeladen werden und anschließend in das Heft eingebunden werden. Die Bilder können eine beliebige Größe annehmen und beliebig im Heft platziert werden. Die Größe kann laufend geändert werden.

\newpage

Dadurch, dass eine digitale Mitschrift besonders in technischen Schulen vorkommt, gibt es ein eigenes Codeelement, mit dem Code Snippets eingebunden werden können. Diese Codezeilen können je nach Programmiersprache oder Scriptsprache verschieden hervorgehoben werden. Das Codeelement unterstützt unter anderem XML, HTML, JAVA, JavaScript, PHP, Python, C, C++, C\# und SQL. Um noch eine bessere Lesefreundlichkeit zu schaffen, sind linksbündig Zeilennummern automatisch integriert, um den Code besser im Überblick zu behalten und eventuelle Referenzierungen anlegen zu können. Dadurch können einzelne Codeabschnitte optimal in die digitale Mitschrift integriert werden.

Es kann vorkommen, dass man einen gewissen Teil aus öffentlichen Heften interressant findet und diesen gerne im eigenen Heft hätte kein. Mithilfe eines Features ist es möglich, Seiten aus öffentlichen Heften in eigene Hefte zu importieren. Dadurch können Inhalte geteilt und anschließend erweitert werden. 

Um bereits niedergeschriebenen Text weiter zu editieren, gibt es eine Funktion, welche die optische Zeichenerkennung (Optical Character Recognition) unterstützt. Mithilfe dieses Features kann der User gedruckten Text in bearbeitbaren Text umwandeln. Durch eine OCR-Engine kann ein Bild auf den Server hochgeladen werden, das nach der Analyse wieder gelöscht wird. Das hochgeladene Bild wird von der integrierten OCR-Engine analysiert und anschließend in ein Textelement ausgegeben. Dieses Element kann nach der Analyse wie ein normales Textelement verwendet werden. Somit kann an bereits bestehendem Text mühelos weitergearbeitet werden.

Gruppenarbeiten oder Referate mit Mehrpersonenteams sind im Schulalltag kaum mehr wegzudenken. Darum gibt es mit DigitalSchoolNotes das Parallel Working System. Mit diesem System ist es möglich, Hefte mit anderen User zu teilen und diese dann gleichzeitig zu bearbeiten. Mit diesem System kann man beliebig viele Nutzer an der Bearbeitung eines Heftes teilhaben lassen. Diese User haben dann die Möglichkeit, geschriebenen Heftinhalt anderer User in Echtzeit zu verfolgen. Durch dieses integrierte System können Gruppenarbeiten optimal durchgeführt werden.

Im Zuge der Umsetzung des Projektes haben sich alle Teammitglieder hinsichtlich ihrer Entwicklungs- und Projektmanagementfähigkeiten maßgeblich weiterentwickelt. Bei der Implementierung wurden moderne Technologien und Tools benutzt. DigitalSchoolNotes bietet Schülern eine effektive und einfache Art ihre digitalen Notizen zu verwalten und setzt einen Grundstein für die Weiterentwicklung von unterstützender Software in diesem Bereich.