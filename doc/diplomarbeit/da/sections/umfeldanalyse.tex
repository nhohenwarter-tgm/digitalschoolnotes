%\section*{Umfeldanalyse}
\cfoot{Niklas Hohenwarter}

Es existiert bereits einiges an Software, welche sich mehr oder weniger dafür eignet digital mitzuschreiben. Der Fokus der Umfeldanalyse wurde auf das gleichzeitige Arbeiten, die Textverarbeitungsfunktion und die Positionierung von Medien gelegt. Die Features der einzelnen Produkte sind großteils bekannt weshalb diese nur oberflächlich beschrieben werden.

\subsubsection{Textverarbeitungsprogramme}
\begin{itemize}
\item \textbf{Microsoft Word:} Am weitesten verbreitete Textverarbeitungssoftware; kostenpflichtig; auf Windows, Mac, Android \& iOS verfügbar; gleichzeitiges Arbeiten teilweise möglich; keine freie Positionierung von Medien
\item \textbf{LibreOffice/OpenOffice Writer:} Textverarbeitung; OpenSource; kostenlos; auf Windows, Mac \& Linux verfügbar; gleichzeitiges Arbeiten nicht möglich; keine freie Positionierung von Medien
\item \textbf{Kingsoft WPS Writer:} Textverarbeitung; kostenlos oder kostenpflichtig; auf Windows, Mac, Android \& iOS verfügbar; sehr stark an Microsoft Word angelehnt; gleichzeitiges Arbeiten nicht möglich; keine freie Positionierung von Medien
\item \textbf{Google Docs:} Textverarbeitung; kostenlos; Betriebssystemunabhängig (Browser); gleichzeitiges Arbeiten möglich; keine freie Positionierung von Medien
\end{itemize}

\subsubsection{Notizverwaltung}
\begin{itemize}
\item \textbf{Microsoft OneNote:} Notizprogramm; Notizen können beliebig platziert und angeordnet werden; Organisation in Notizbüchern; gleichzeitiges Bearbeiten möglich; auf Windows, Mac, Android \& iOS verfügbar; kostenlos
\end{itemize}