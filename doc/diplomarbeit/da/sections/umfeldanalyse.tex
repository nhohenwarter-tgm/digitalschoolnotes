%\section*{Umfeldanalyse}
\cfoot{Niklas Hohenwarter}

Es existiert bereits einiges an Software welche sich mehr oder weniger dafür eignet digital mitzuschreiben. Die Features der einzelnen Produkte sind großteils bekannt, deshalb werden diese nur oberflächlich beschrieben:

\begin{itemize}
\item \textbf{Microsoft Word:} Am weitesten verbreitete Textverarbeitungssoftware; kostenpflichtig; auf Windows, Mac, Android \& iOS verfügbar
\item \textbf{Libre Office Writer:} Textverarbeitung; OpenSource; kostenlos; auf Windows, Mac \& Linux verfügbar
\item \textbf{OpenOffice Writer:} Textverarbeitung; OpenSource; kostenlos; auf Windows, Mac \& Linux verfügbar
\item \textbf{Kingsoft WPS Writer:} Textverarbeitung; kostenlos oder kostenpflichtig; auf Windows, Mac, Android \& iOS verfügbar; sehr stark an Microsoft Word angelehnt
\item \textbf{Microsoft OneNote:} Notizprogramm; Notizen können beliebig platziert und angeordnet werden; Organisation in Notizbüchern; gleichzeitiges Bearbeiten möglich; auf Windows, Mac, Android \& iOS verfügbar; kostenlos
\item \textbf{Google Docs:} Textverarbeitung; kostenlos; Betriebssystemunabhängig (Browser); gleichzeitiges Arbeiten mögliich
\end{itemize}