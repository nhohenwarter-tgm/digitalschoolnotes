%\section*{User und Rollenmanagement}
\cfoot{Philipp Adler}

\subsubsection{Usermanagment}
Unter dem Begriff Usermanagement versteht sich, dass Erzeugen, Verwalten und Löschen von Benutzerkonten. Es soll dazu dienen, jedem registrierten User eindeutig zu identifizieren und kontrollieren, z.B. ob die monatliche Rate für den Pro-Account überwiesen wurde. Außerdem soll die bereits in Anspruch genommene Speicherkapazität überwacht werden. Dazu ist es notwendig, dass jeder DSN-User, durch eine Kombination von Daten differenzierbar ist.
\subsubsection{Authentisierung}
\paragraph{Registrierung}
Unter Authentisierung versteht sich der Nachweis der behaupteten Identität. Im Falle von DSN handelt es sich hierbei um die eindeutige Email-Adresse, welche einmalig im System verwendet wird. Identität bedeutet Gewissheit/Sicherheit, von wem die Information stammt. Dadurch kann jedes Handeln jemanden zugewiesen werden. Im Falle, dass unsere Richtlinen nicht eingehalten werden, kann DSN sofort reagieren.\\

Ein weiterer Identitätspunkt wäre, dass geheime Passwort, welches aus Sicherheitsgründen mindestens 8 Zeichen beinhalten muss. Durch 8 Zeichen wird versucht, Cyberkriminellen das Knacken von Passwörtern erschweren. Zum Abschluss der Registrierung müssen noch die Nutzungsbedingungen akzeptiert werden.\\
\grqq{}Allgemeine Geschäftsbedingungen (AGB) sind vertragliche Klauseln, die zur Standardisierung und Konkretisierung von Massenverträgen dienen. Sie werden von einer Vertragspartei einseitig gestellt und bedürfen daher einer bes. Kontrolle, um ihren Missbrauch zu verhindern.\grqq{}\cite{AGB}\\
\cite{VERTEILTE_SYSTEME}\cite{PASSWORT_SCHUTZ}\\

\insertpicture{images/usermanagement/Registrierung}{Authentisierung bei DSN}{(selfmade)}{itm:authentisierung-chart}{0.20}

Ein Captcha(Completely Automated Public Turing test to tell Computers and Humans Apart) bei der Registrierung soll verhindern, dass sich eine Software oder ein Bot registrieren kann.\\
Verallgemeinert, ein Captcha dient der Sicherheit und hat die Aufgabe jede Eingabe auf ihre Herkunft zu prüfen. Um ein Captcha anzuzeigen, muss eine JavaScript Library eingebunden werden \cite{CAPTCHA}.
\begin{lstlisting}[caption={Einbindung der JS-Library Recaptcha}]
<script src="https://www.google.com/recaptcha/api.js?
onload=vcRecaptchaApiLoaded&render=explicit" async defer></script>
<script src="https://cdnjs.cloudflare.com/ajax/libs/angular-recaptcha/2.2.5/
angular-recaptcha.min.js"></script>
\end{lstlisting}

Bei jeder Registrierung generiert sich das Captcha neu, wodurch es nur einmal eingelöst werden kann.\\

\insertpicture{images/usermanagement/Captcha}{Auslösen des Captchas}{(selfmade)}{itm:captcha-chart}{0.35}

\paragraph{Datenbank}
Nach der Authentisierung wird der/die zukünftige BenutzerIn nach dem im Kapitel 5.3.2.1 beschriebenen Datenmodell in die Datenbank persistiert. Der Registierungsprozess ist aber erst nach einlösen des Validierungstoken abgeschlossen. Bis dahin, ist dem/der BenutzerIn nicht erlaubt, sich auf DSN anzumelden. \\

Es kann auch sein, dass sich ein bereits registrierter Account sich nicht anmelden kann. Dies könnte z.B. der Fall sein wenn jemand gegen die Regeln und Gesetze von DSN verstößt und daraufhin gesperrt wird. Der User hat keine andere Wahl als Kontakt zum Administrator aufzunehmen.

\paragraph{Email}
Um den Registrierungsprozess zu beenden, wird dem nahestehenden User ein Token per Mail von unserem Mailserver zugesandt. Dieser Token dient der Identifizierung und Authentifizierung und könnte folgendermaßen aussehen:
\begin{lstlisting}[caption={Validierungstoken für die Aktivierung des DSN-Accounts}]
http://digitalschoolnotes.com/validate/
\\dad9574635aad7d6549536db38f7839c042f7704b3bd74acc427f075d0601470
\end{lstlisting}

Bei der Erstellung eines solchen Tokens wird die Email-Adresse des Benutzers und das aktuelle Datum kombiniert. Die Email-Adresse dient dazu, um zu wissen, welcher Account aktiviert werden soll. Die beiden Werte werden miteinander verknüpft und in einen Hash umgewandelt. Dieser Hash dient als Aktivierungstoken.\\
Ohne Token, könnte sich jeder bei DSN registrieren, ohne dafür eine valide Email-Adresse haben zu müssen.\\

Dem Token wird mittels Datum eine, Lebensdauer zugeteilt. Falls dieser Hash nicht innerhalb von 24 Stunden eingelöst wird, verfällt dieser Token und es muss vom User neu angefordert werden. 

\paragraph{Anmeldung}
Hat der Benutzer den Registrierprozess erfolgreich abgeschlossen steht ihm/ihr jetzt frei, sich anzumelden. Beim Login meldet sich der User anhand seiner Email-Adresse und Passwort an. Oder er nutzt die Möglichkeit sich mittels OAuth anzumelden.\\

OAuth steht für Open Authentication und bietet dem Nutzer die Möglichkeit Daten über einen Webservice auszutauschen.„OAuth sichert die Programmschnittstelle von Web-Anwendungen und verwendet für die Übertragung der Nutzeridentifikation dessen Passwort und einen Token“\cite{OAUTH}. Beim Zugriff auf sensible Daten muss der Benutzer keine zusätzlichen Information und auch keine Identität preisgeben. Der Provider holt sich die Benutzerdaten von Facebook oder Google+ und erstellt für den User einen Account. Sozusagen besteht für unsere Endanwender die Möglichkeit, den vorher beschriebenen Registierungsprozess auszulassen und sich direkt über OAuth zu registrieren und anzumelden.

\insertpicture{images/usermanagement/Anmelden}{Klassische Anmeldung oder mittels OAuth}{(selfmade)}{itm:login-chart}{0.35}

Beim Anmelden wird überprüft, ob die übergebene Email-Adresse in der Datenbank existiert. Ist Sie keinem Benutzer zugeordnet, existiert der Benutzer nicht und die Authentifikation schlägt fehl. Andernfalls wird das eingegebene Passwort überprüft. Sollte dieses korrekt sein, wird der Anwender in den Userbereich weitergeleitet.\\

Surft ein Benutzer auf unserer Seite, arbeitet im Hintergrund ein Session Timeout, welches die Aktivtät des eingeloggten Benutzer überprüft.\\
Im Falle einer Inaktivität von 1 Stunde wird der Anwender automatisch abgemeldet und zur Anmeldeseite weitergeleitet. Es dient als Vorbeugemaßnahmen und kontrolliert unauthorizierte Aktivtäten.

\begin{lstlisting}[caption={Session Timeout}]
try:
    user = User.objects.get(email='exampleATexample.com')
except:
    user = None
if user is not None and user.check_password('myPassword'):
    user.backend = 'mongoengine.django.auth.MongoEngineBackend'
    login(request, user)
    request.session.set_expiry(60 * 60 * 1) # 1 hour timeout
\end{lstlisting}
\newpage 

\subsubsection{Userbereich}

\insertpicture{images/usermanagement/Usersicht}{Navigationsleiste als User}{(selfmade)}{itm:navigation-chart}{0.85}

In der Navigationsleiste befinden sich:
\begin{itemize}
\item \textbf{Stundenplan}\\ Im Stundenplan kann jeder seine Schulstunden und Schulfächer manuell eintragen. Im Hintergrund werden diverse Fächer mit einem Schulheft verknüpft.
\item \textbf{Hefte}\\ Hier werden die Schulhefte aufgelistet, welche während der Schulstunden zum Einsatz kommen.
\item \textbf{Kontoeinstellungen}\\ Die Kontoeinstellungen sind in 3 Bereiche gegliedert.
\begin{enumerate}
\item \textbf{User-Daten bearbeiten}\\ Ändern der Benutzerinformationen
\item \textbf{Passwort ändern}\\ Überschreiben des alten Passworts
\item \textbf{Account löschen}\\ Alle angelegten Hefte, sowie der DSN-Account werden gelöscht.
\end{enumerate}
\item \textbf{Suche}\\ Jede/r BenutzerIn hat die Möglichkeit, nach Freunden oder anderen registrierten Anwendern, mittels Vorname, Nachname oder Email-Adresse, zu suchen. Auf dessen Profil, ist zum Einen der volle Name, die Email-Adresse und die Berechtigungsstufe, ob Standardbenutzer, Pro-Benutzer oder Administrator, zu sehen.\\
Ein weiterer wichtiger Punkt sind die öffentlichen Hefte. Jedem User steht es frei, von öffentlichen Heften, einzelne Heftseiten in seine Eigenen zu exportieren.
\item \textbf{Abmelden}\\ Abmelden des Accounts
\end{itemize}
\newpage 

\subsubsection{Berechtigungen}
Im System befinden sich drei verschiedene Berechtigungsstufen, welche sind: der Standard-Benutzer, Pro-Benutzer und Administrator. Jede/r registrierte AnwenderIn ist zu Beginn ein Standard-Benutzer.\\

\begin{itemize}
\item \textbf{Standard-Benutzer}\\ Der Standard-Benutzer soll dazu dienen, DSN besser kennenzulernen und auf den Geschmack von den tollen Features, wie OCR oder PWS zu kommen. Unser System bietet dafür eine Testdauer von 90 Tagen an und steht in Form von digitalen Heften in begrenzter Stückzahl zur Verfügung.
\item \textbf{Pro-Benutzer}\\ In Zukunft wollen wir Standard-Benutzern die Möglichkeit bieten, durch eine geringe monatliche Zahlung, seinen Account auf einen Pro-Account upzugraden. Dadurch werden dem/r SchülerIn erweiterte Funktion angeboten wie eine unbegrenzte Heftanzahl, keine Werbung, sowie keine Speicherbeschränkung. Das derzeitige Limit liegt bei 1GB.\\
Das Konzept gibt es bereits, aber die Umsetzung ist noch in der Entwicklungsphase.
\item \textbf{Administrator}\\ Die letzte Berechtigungsstufe sind Administratoren. Sie sind ebenfalls Pro-User, haben aber im Gegensatz zu Ihnen einen eigenen Admin-Bereich. Dort werden sämtliche Daten über Benutzer aufgelistet. Dieser Bereich kann mit /admin nach der URL aufgerufen werden. Er unterscheidet sich durch den schwarzen Menübalken.
\end{itemize}

\subsubsection{Adminsicht}
\insertpicture{images/usermanagement/Adminsicht}{Navigationsleiste als Admin}{(selfmade)}{itm:navigation-admin-chart}{0.85}
\paragraph{User Management}
Auf der User Management Page werden alle Benutzer von DSN aufgelistet. Der Admin hat Einsicht auf die Email-Adresse, Vorname, Nachname und auf die Berechtigungsstufe.

Dem Administrator steht es frei, andere Benutzer zu löschen, deren Berechtigungsstufe zu ändern oder sie mittels Mail, auf etwas hinzuweisen. Neben der Auflistung aller Benutzer, kann auch im Bedarfsfall nach einzelnen Nutzern gesucht werden. Die Sucheingabe wird mit Vornamen, Nachnamen und der Email-Adresse verglichen. Zusätzlich besteht die Möglichkeit die Tabelle in alphabetischer Reihenfolge zu sortieren.

\insertpicture{images/usermanagement/Usermanagment}{Usermangement-Page}{(selfmade)}{itm:navigation-admin-chart}{0.85}

Unsere Usermanagement-Page wurde so designt, dass durch die kurze Ladezeit, dem Administrator die Seite schnell zur Verfügung steht. Die Performancesteigerung ergibt daraus, dass nur eine bestimmte Anzahl an DSN-Usern in Form einer Tabelle vom Server geladen werden. Ist das Ende einer Tabellenseite erreicht, kann auf die nächste Seite umgeblättert werden. Somit können alle User aufgelistet werden.

\paragraph{Benachrichtigung}
Falls Pro-User, die den Zahlungen nich nachkommen, hat der Administrator von DSN das Recht, den User zu löschen. DSN gibt dem User noch die Möglichkeit, seine Daten bzw. Hefte zu sichern bevor er gelöscht wird. Der zu löschende Benutzer empfängt rechtzeitig eine Email, wo darauf hingewiesen wird, dass sein Account und alle dazugehörigen Daten nach 7 Tagen gelöscht werden. Am Server von DSN läuft ein Cron-Daemon, welcher jeden Tag überprüft, wann der zu löschende User entfernt werden soll. \grqq{}Der Cron-Daemon ist ein Dienst, der automatisch Skripte und Programme zu vorgegebenen Zeiten starten kann.\grqq{}\cite{CRON}\\
Im Falle, dass jemand länger als 3 Monate interaktiv ist, wird ihm ebenfalls eine Informationsmail zugesendet. Der User wird erinnert, dass er sich binnen 7 Tagen einlogen muss, sonst wird der Account mit alle den Daten gelöscht. \cite{COMMANDS}\cite{CRON}

\begin{lstlisting}[caption={Cronjob für die {\"U}berpr{\"u}fung der Inaktivt{\"a}t und L{\"o}schung}]
# m h  dom mon dow   command
# * *   */1   *   *    python3 /home/stable/dsn/manage.py inform
# * *   */1   *   *    python3 /home/stable/dsn/manage.py delete
\end{lstlisting}