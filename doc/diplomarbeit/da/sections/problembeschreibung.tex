%\section*{Problembeschreibung}
\cfoot{Niklas Hohenwarter}

Im Schulunterricht mitzuschreiben ist wichtig, jedoch immer weniger Schüler machen dies auch. Falls sie doch mitschreiben, dann verschwindet der Zettel oder Collegeblock meistens oder wird nie wieder angesehen. Es ist kompliziert den in immer größeren Teilen digital ablaufenden Unterricht auf Papier festzuhalten und seine Mitschriften zu organisieren. Als Resultat dieser Umstände stirbt die händisch geführte Mitschrift langsam aber sicher aus. \\

Statt der händischen Mitschrift wird die digitale Mitschrift am im Unterricht verwendeten Laptop immer beliebter. Diese hat allerdings bis heute ähnliche Probleme. Es ist kompliziert die Struktur des Tafelbildes in z.B. ein Word-File zu übernehmen. Außerdem passiert es häufig, dass die Datei welche die Mitschrift enthält nicht mehr gefunden wird oder einfach einen unpassenden Dateinamen hat, welcher nicht die entsprechende Mitschrift vermuten lässt. Mit aktuellen Produkten ist es ebenfalls mühsam digital mitzuschreiben.