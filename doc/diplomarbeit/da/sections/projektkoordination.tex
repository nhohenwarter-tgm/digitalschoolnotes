%\section*{Projektkoordination}
\cfoot{Adin Karic}

Eine gut durchdachte Projektkoordination ist die Voraussetzung für die erfolgreiche Umsetzung eines Projektes. Dabei kann man generell zwischen zwei Arten des Projektmanagements unterscheiden: dem klassischen Projektmanagement und dem agilen Projektmanagement.

Das klassische Projektmanagement zeichnet sich durch am Projektbeginn klar formulierte Ziele und die Unterteilung des Projektes in sogenannte Meilensteine und weitgehend voneinander abgegrenzte Phasen aus.

\insertpicture{images/Projektkoordination/vmodell.png}{Phasen des klassischen V-Modells}{\cite{VMODELL}}{itm:vmod}{1.0}

\newpage

Das agile Projektmanagement geht davon aus, dass die meisten Softwareentwicklungsprozesse zu komplex sind, um diese auf klassische Art und Weise zu planen. Agile Ansätze zählen also auf die Dynamik der verschiedenen Projektparameter, wie zum Beispiel das Umfeld oder die Projektziele. Einige bekannte Vorgehensweisen des agilen Projektmanagements sind SCRUM, TDD (test-driven development) oder XP (Extreme Programming). \cite{SCRUM}

Alle agilen Entwicklungsmethoden haben aber eine gemeinsame Basis: das agile Manifest. Das agile Manifest ist eine Ansammlung von agilen Prinzipien, welches erstmals 2001 einer breiten Öffentlichkeit zugänglich wurde. In dem Manifest legen die damaligen Erstunterzeichner die agilen Handlungsgrundsätze für eine erfolgreiche agile Softwareentwicklung fest.

\textit{,,Wir erschließen bessere Wege, Software zu entwickeln, indem wir es selbst tun und anderen dabei helfen. Durch diese Tätigkeit haben wir diese Werte zu schätzen gelernt:
\begin{itemize}
\item Individuen und Interaktionen mehr als Prozesse und Werkzeuge
\item Funktionierende Software mehr als umfassende Dokumentation
\item Zusammenarbeit mit dem Kunden mehr als Vertragsverhandlung
\item Reagieren auf Veränderung mehr als das Befolgen eines Plans
\end{itemize}
Das heißt, obwohl wir die Werte auf der rechten Seite wichtig finden, schätzen wir die Werte auf der linken Seite höher ein." } \cite{AGMAN}

Mit der Annahme, dass Softwareentwicklungsprozesse empirische Prozesse sind, wurde für die Durchführung des Projektes ,,Digital School Notes'' das agile Projektmanagement gewählt. Da die Softwareumgebung selten vollständig definiert ist, die Anforderungen veränderlich sind und das Wissen und Können für den besten Lösungsansatz oft unvollständig ist, wird ein agiles Vorgehen einem klassischen vorgezogen. Konkret wurde für die Projektkoordination die agile Projektmanagementmethode SCRUM angewendet. \cite{SCRUM}

\insertpicture{images/Projektkoordination/unterschiede.png}{Klassisches PM vs. Agiles PM}{\cite{UNT}}{itm:unters}{0.8}

\subsubsection{Kurzeinführung in Scrum}
Scrum bedeutet übersetzt ,,Gedränge'' und stellt ursprünglich einen Spielzug im Rugbysport dar. Mit der Analogie zum ,,Gedränge" wollten die beiden Ideengeber  Ikujirō Nonaka und H. Takeuchi darauf anspielen, dass kleine, vernetzte Teams die besten Resultate liefern. Scrum ist eine Umsetzung des ,,Lean Developments" (schlanke Entwicklung) für das Projektmanagement. \cite{SCRUM}
\paragraph{Rollen in Scrum}
Scrum definiert als agiler Ansatz drei unterschiedliche Rollen: den Product Owner, den Scrum Master und das Entwicklungsteam.

\textbf{Product Owner}\\
Der Product Owner ist hauptsächlich für die zu implementierenden Funktionen, sowie den wirtschaftlichen Erfolg des Produkts verantwortlich. Er entscheidet maßgeblich, welche Eigenschaften das Produkt aufweist und welche nicht. Dabei stehen die Befriedigung der Bedürfnisse der Stakeholder, sowie der maximale wirtschaftliche Nutzen des Produktes im Mittelpunkt. Der Product Owner erstellt, priorisiert und entscheidet über die Produkteigenschaften und schließlich wann diese umgesetzt werden. Zur Festlegung dieser Eigenschaften erstellt der Product Owner ein Product Backlog. Eine weitere wichtige Aufgabe des Product Owners ist der ständige Kontakt mit den Stakeholdern des Produkts, sowie die Vertretung der Interessen selbiger. \cite{SCRUM}

\textbf{Scrum Master}\\
Der Scrum Master ist für die erfolgreiche Umsetzung von Scrum als Projektmanagementmethode verantwortlich. Er definiert die Regeln des Scrum und achtet auf deren Einhaltung. In enger Zusammenarbeit mit dem Entwicklungsteam ist er dafür verantwortlich, dass Hindernisse, die eine erfolgreiche Umsetzung gefährden könnten, aus dem Weg geräumt werden. Der Scrum Master stellt gegenüber dem Entwicklungsteam eine ,,dienende", situative Führungskraft dar und erteilt selbst keine Arbeitsanweisungen. \cite{SCRUM}

\textbf{Entwicklungsteam}\\
Das Entwicklungsteam ist in erster Linie für die Implementierung der vom Product Owner festgelegten Produktfunktionalitäten zuständig. Ein Entwicklungsteam besteht meistens aus drei bis neun Mitgliedern. Es organisiert sich weitgehend selbst und hält die Qualitätsstandards aufrecht. Eine interdisziplinäre Besetzung des Entwicklungsteams ist essentiell, um die in den Sprints anstehenden Aufgaben möglichst unabhängig voneinander lösen zu können. Die Teammitglieder selbst sind neben speziellen Fähigkeiten aber auch Generalisten, da sie bei Ausfallen eines Teammitglieds dessen Arbeit fortführen können müssen. Eine entsprechende Dokumentation sollte hier hilfreich sein. Eine wichtige Aufgabe des Entwicklungsteams ist zudem die Schätzung des Aufwands der verschiedenen Einträge im Product Backlog. Dies wird üblicherweise mit Story Points realisiert, welche für eine bestimmte Zeitdauer stehen. \cite{SCRUM}

\paragraph{Artefakte in Scrum}
\textbf{Product Backlog}\\
Das Product Backlog stellt eine Menge an Anforderungen an das zu entwickelnde Produkt dar. Es ist keineswegs statisch und kann ständig verändert werden. Für die Gestaltung des Product Backlogs ist der Product Owner zuständig. Es stellt den Ausgangspunkt für die umzusetzenden Funktionalitäten des Produkts dar. Diese Anforderungen sind jedoch nicht technisch, sondern eher anwenderfreundlich als sogenannte User Stories definiert. In so einer User Story wird in meist einem knappen Satz der Nutzen der Funktionalität für den Endnutzer dargestellt. Ein Beispiel für eine solche User Story wäre: ,,Als Benutzer möchte ich mich mittels Facebook am System anmelden können". \cite{SCRUM}

\textbf{Release Backlog}\\
Das Release Backlog fasst alle User Stories zusammen, welche in einem bestimmten Release (Version) des Produkts implementiert werden sollen. Die User Stories stammen aus dem Product Backlog und folglich kann das Product Backlog in mehrere Release Backlogs unterteilt werden. \cite{SCRUM}

\textbf{Sprint Backlog}\\
Das Sprint Backlog definiert die für einen bestimmten Sprint umzusetzenden Produktfunktionalitäten (diese stammen aus einem bestimmten Release Backlog). Das Sprint Backlog wird laufend von den Teammitgliedern aktualisiert, also zum Beispiel wenn eine User Story umgesetzt wurde. Aus den verschiedenen Parametern eines Sprints lässt sich schließlich ein Burndown chart für jeden Sprint generieren. \cite{SCRUM}

\textbf{Defect Backlog}\\
Im Defect Backlog werden mögliche Fehler oder Hindernisse bei der Implementierung der User Stories festgehalten. Es dient dazu sich einen Überblick über die verschiedenen Arten und das Aufkommen von Bugs zu machen. Ein Sprint wird erst dann abgenommen, wenn alle User Stories des betreffenden Sprints weitgehend fehlerfrei sind. \cite{SCRUM}

\textbf{Impediment Backlog}\\
Ähnlich wie beim Defect Backlog besteht das Impediment Backlog aus Hindernissen. Diese Hindernisse sind aber keine technischen Fehler, sondern meist organisatorische Hindernisse oder jene, welche das Projektmanagement betreffen. Für die Erstellung des Impediment Backlogs, sowie für die Beseitigung der Hindernisse, ist der Scrum Master verantwortlich. \cite{SCRUM}

\textbf{Sprint-Burndown-Chart}\\
Das Sprint-Burndown-Chart dient zur Visualisierung des Arbeitsfortschritts im aktuellen Sprint. Es wird nach Erledigen der User Stories im Sprint laufend aktualisiert und kann dem Team und dem Product Owner gefährliche Verzögerungen zeigen. Dabei ist auf der Ordinate der (Rest-)Aufwand in Stunden (oder eben Story Points) und auf der Abszisse die Zeit aufgetragen. Mit der Erledigung von User Stories sinkt der Restaufwand und es vergeht natürlich Zeit. Dadurch lässt sich der tatsächliche Fortschritt sowie der prognostizierte Termin für das Fertigstellen aller User Stories (Restaufwand = 0) darstellen.\cite{SCRUM}

\insertpicture{images/Projektkoordination/burndownErk.png}{Beispiel eines Sprint-Burndown-Charts}{\cite{BURNERK}}{itm:burnerk}{1.0}

\textbf{Sprint-Endbericht}\\
Immer wenn ein Sprint beendet wird, wird ein Sprint-Endbericht vom Product Owner erstellt. Dieser umfasst den Sprint-Verlauf, also die umgesetzten Funktionalitäten, die Hindernisse, sowie das Sprint-Burndown-Chart.

\paragraph{Scrum-Ablauf}
Scrum beginnt mit der Festlegung der verschiedenen Rollen. Es werden also Product Owner, Scrum Master, sowie das Entwicklungsteam bestimmt. Der nächste Schritt ist das Erstellen und Sammeln der sogenannten User Stories. Diese User Stories stellen im Grunde die Funktionen des zukünftigen Produktes dar. Aus diesen User Stories wird nun ein Product Backlog definiert. Dabei entscheidet der Product Owner maßgeblich, welche User Stories hineinkommen und welche gestrichen werden. Der Product Owner und das Enwicklungsteam schätzen dann den Aufwand der verschiedenen User Stories in Story Points oder Zeiteinheiten. Als Nächstes priorisiert der Product Owner, welche Funktionalitäten wann (in welchem Sprint) umgesetzt werden. \cite{SCRUM}

\newpage

Aus dieser Priorisierung ergibt sich nun das Sprint Planning (ein Release Planning davor ist optional). Das Ergebnis des Sprint Planning ist das Sprint Backlog. Es enthält alle funktionalen Anforderungen, die in diesem bestimmten Sprint umgesetzt werden sollen. \cite{SCRUM}

Ein Sprint dauert normalerweise zwei bis vier Wochen und stellt den Zeitraum dar, in dem die Anforderungen aus dem jeweiligen Sprint Backlog umgesetzt werden sollen. Innerhalb des Sprints findet täglich ein kurzes Daily Scrum Meeting statt. Das Daily Scrum Meeting ist ein Stand-up-Meeting und dient zum Informationsaustausch zwischen Entwicklungsteam, Product Owner und Scrum Master. Bei jedem Daily Scrum Meeting werden prinzipiell diese drei Fragen gestellt:
\begin{itemize}
\item Was haben Sie seit dem letzten Meeting getan?
\item Was werden Sie heute machen?
\item Hatten Sie etwaige Probleme?
\end{itemize}
\cite{SCRUM}

Durch diese Fragen möchte der Scrum Master ein Maximum an Information gewinnen. Durch das kurze Meeting halten sich alle Beteiligten über den Projektfortschritt am Laufenden und der Scrum Master hält mögliche Probleme fest und versucht diese zu lösen.

Am Ende eines Sprints steht die Sprint Review. Sie stellt die Abnahme der in diesem Sprint implementierten Funktionalitäten dar. Der Product Owner nimmt, meist gemeinsam mit dem Auftraggeber, diese Funktionen ab und erstellt daraus den Sprint-Endbericht. \cite{SCRUM}

Nach dem Sprint Review erfolgt die Sprint Retrospektive. Hierbei geht es nicht um die umgesetzten Anforderungen sondern allein um das Feedback des Teams hinsichtlich der Zusammenarbeit und Organisation. Dabei nennen die Mitglieder positive und negative Aspekte des Sprints. Abschließend wird zu jedem negativen Punkt eine Verbesserungsmaßnahme definiert. \cite{SCRUM}

Sollte es der Fall sein, dass alle User Stories aus dem Product Backlog abgearbeitet sind, findet nach diesem Sprint eine Endabnahme statt und das fertige Produkt liegt vor. Ansonsten wird das Ganze wiederholt und ein neuer Sprint mit neuem Sprint Backlog wird gestartet. Das Ergebnis eines jeden Sprints ist ein Produktinkrement, also eine verbesserte Version des Produktes. \cite{SCRUM}

\insertpicture{images/Projektkoordination/scrumablauf.png}{Scrum-Ablauf}{\cite{SCRUMABL}}{itm:scrumabl}{1.0}

\paragraph{Vorteile von Scrum}
Scrum bringt, als agiler und schlanker Ansatz des Projektmanagements, einige Vorteile mit sich, weswegen sich das Projektteam für Scrum entschieden hat. Durch Scrum erhält der Auftraggeber schon sehr früh einen Einblick in den Entwicklungsstand und kann daher mit einer Änderung von Anforderungen oder Ähnlichem auf die Produktentwicklung Einfluss nehmen. Das ist einer der großen Vorteile von Scrum. Der Auftraggeber weiß zu jeder Zeit, wie es mit dem Projektfortschritt steht und kann gegebenenfalls eingreifen, falls etwas schief läuft.\cite{SCRUM}

Ein weiterer Vorteil ist die aktive Miteinbeziehung des Auftraggebers, was in meist zufriedenstellender Software mündet.
Im Gegensatz zu der statischen, langen Planung einer klassischen Vorgehensweise, ist eine Änderung der Anforderungen bei Scrum jederzeit möglich. Zudem steigt die Produktivität, weil nur die wirklich effektiv benötigten Funktionen umgesetzt werden. Außerdem erhöht Scrum, durch die selbstorganisierten Teams, maßgeblich die Motivation der Teammitglieder. \cite{SCRUM}

\newpage

\subsubsection{Scrum im Team}
Das Diplomprojektsteam hat sich für die Durchführung des Projektes für den agilen Ansatz Scrum entschieden. Die Gruppe zählte fünf Personen. Niklas Hohenwarter wurde auf der Projektwoche in Kärnten vom Team zum Product Owner gewählt. Selina Brinnich fungierte als technische Architektin. Das Entwicklungsteam bestand aus Philipp Adler, Adin Karic und Thomas Stedronsky, wobei hier der Begriff Entwicklungsteam nicht zu wörtlich zu nehmen ist, denn es waren alle fünf Teilnehmer aktiv an der Entwicklung des Produktes beteiligt. Als Scrum Master stellte sich Prof. Borko zur Verfügung.

Die Projektwoche in Kärnten im Oktober stellte den Kick Off für das Projekt dar. Das Team schrieb wichtige Dokumente und vergab Rollen dort. Auch die Erstellung der User Stories im Team wurde dort erledigt. Auf mögliche Akzeptanzkriterien und Tasks (Unteraufgaben einer User Story) wurde Rücksicht genommen. Die User Stories wurden schließlich in dem Online-Scrum-Tool Ontime von Axosoft eingefügt. Ontime wurde auch für den weiteren Verlauf des Projekts als Unterstützung für den Scrum-Prozess verwendet. Das Team einigte sich auf eine Sprintlänge von drei Wochen.

\insertpicture{images/Projektkoordination/userst.png}{Beispiel für eine User Story in OnTime}{(selfmade)}{itm:userst}{1.0}

Das Team hat sich feste Arbeitszeiten ausgemacht und kam so auf eine durchschnittliche Arbeitszeit von 13 Stunden pro Woche. Zur Steigerung der Produktivität wurde die sogenannte Pomodoro-Technik \cite{POMOT} angewandt. Diese Technik besagt, dass man durch häufige, kleine Pausen viel effizienter arbeiten kann. Die Arbeitseinheiten werden als Pomodori (Einzahl Pomodoro) bezeichnet. In unserer Anwendung der Technik dauerte ein Pomodoro 25 Minuten. Nach jedem Pomodoro wurde eine fünfminütige Pause eingelegt. Nach jedem vierten Pomodoro gab es eine lange Pause von 15 Minuten.

\insertpicture{images/Projektkoordination/pomodoro.png}{Ansicht eines Pomodoro-Timers}{(selfmade)}{itm:pomo}{0.8}

Die User Stories wurden vom Product Owner priorisiert und gemeinsam mit dem Team wurde der Aufwand in Story Points geschätzt. In unserer Kalkulation entsprach ein Story Point einer Arbeitszeit von vier Stunden. 

Durch die Priorisierung wurden die ersten User Stories in ein Sprint Backlog gepackt. Der Evaluierungssprint diente hauptsächlich der Evaluierung von Tools und Technologien, speziell jener, die in den jeweiligen Spezialgebieten der Mitglieder von Relevanz sein könnten. Das Ergebnis dieses Sprints war die Implementierung von funktionstüchtigen Prototypen, die zu einem späteren Zeitpunkt des Projektes eine wichtige Rolle einnehmen würden.

\insertpicture{images/Projektkoordination/sprint0.png}{Burndown-Chart des Evaluierungssprints}{(selfmade)}{itm:bur}{0.8}

Im ersten Implementierungssprint ging es vor allem um die Implementierung der System-Basis für unser Projekt. Dabei wurden Funktionen wie das Registrieren und Anmelden der Nutzer, die Profile der Nutzer, einfache Hefterstellung, allgemeine Benutzerverwaltung, sowie ein einfacher Stundenplan umgesetzt.

\insertpicture{images/Projektkoordination/sprint1.png}{Burndown-Chart des ersten Sprints}{(selfmade)}{itm:bur}{0.8}

\newpage

Der zweite Sprint befasste sich hauptsächlich mit der Umsetzung des Stundenplans sowie den allgemeinen Kontoeinstellungen der Nutzer. Zusätzlich wurde in diesem Sprint an trivialen Heftoperationen wie zum Beispiel dem Löschen von Elementen gearbeitet.

\insertpicture{images/Projektkoordination/sprint2.png}{Burndown-Chart des zweiten Sprints}{(selfmade)}{itm:bur}{0.8}

Im dritten Sprint ging es meist um die Heftfunktionen und die Heftansicht. In diesem Sprint wurde die Ansicht des Heftes neu gestaltet, das Blättern in Heften wurde ermöglicht und das Importieren von Heftseiten aus anderen Heften wurde implementiert.

\insertpicture{images/Projektkoordination/sprint3.png}{Burndown-Chart des dritten Sprints}{(selfmade)}{itm:bur}{0.8}

\newpage

Die Planung des vierten Sprints war die Folge einer Restrukturierung und der Neupriorisierung des Product Owners. Aufgrund der Komplexität der Implementierung der JS-Frameworks in das bestehende System ist hier zusätzlicher Arbeitsaufwand entstanden. Oberste Maxime war die Fertigstellung der drei Heftelemente: das Textelement, das Programmcodeelement und das Bildelement.

Der letzte Sprint befasste sich vor allem mit zwei Funktionen des Produktes. Das gemeinsame Arbeiten (Parallel Working System) und das OCR-Modul (optische Zeichenerkennung) wurden umgesetzt. Zudem wurden allfällige Mängel und Fehler behoben.

\insertpicture{images/Projektkoordination/burnges.png}{Burndown-Chart des gesamten Projektes}{(selfmade)}{itm:bur}{0.8}

Nach jedem Sprint fand eine Sprint-Abnahme durch den Product Owner und den Auftraggeber statt. Dabei wurden die funktionalen Anforderungen, die in dem jeweiligen Sprint umzusetzen waren, überprüft und abgenommen. Nach der Sprint-Abnahme erfolgte das Sprint-Retrospektive-Meeting in dem positive und negative Aspekte der Zusammenarbeit und Organisation in diesem Sprint diskutiert wurden. Der letzte Schritt stellte die Formulierung von Verbesserungsvorschlägen und deren Umsetzung für die folgenden Sprints dar.

Insgesamt ist zu sagen, dass die Projektmanagementmethode Scrum erfolgreich umgesetzt wurde und der Fertigstellungstermin sowie die umgesetzten Produktfunktionen der Planung entsprachen.
