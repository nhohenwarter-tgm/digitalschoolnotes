%\section*{Projektidee}
\cfoot{Adin Karic}
Die Idee hinter dem Diplomprojekt ,,Digital School Notes" war es eine Web-Applikation zur Führung einer digitalen Mitschrift zu entwickeln. Der Zugriff und das Bearbeiten soll von Desktop Systemen, Laptops und Tablets über eine Website möglich sein. Des Weiteren soll die Applikation mit Ausnahme der Heftbearbeitung auch auf Handys verfügbar sein. Das System sollte eine hohe Verfügbarkeit und gleichzeitig, für Schüler optimiert, eine leichte Bedienung bieten. 

Es sollte ein System entwickelt werden das extra auf die Bedürfnisse von Schülern angepasst ist. Das Produkt soll es ermöglichen, ganz einfach und schnell verschiedene externe Medien wie zum Beispiel Bilder, Links auf andere Webseiten oder Code-Snippets in ein ,,Heft" einzufügen und zu verwalten. Durch eine sehr simpel gestaltete Benutzeroberfläche sollte die Bedienung zum Einfügen solcher Medien, im Vergleich zu anderen Produkten auf dem Markt, deutlich vereinfacht werden. Das Anbieten von verschiedenen Dateiformaten sowie der sichere Umgang mit den Daten der Nutzer war ebenso ein Anliegen.

Durch einen eigenen Stundenplan sollte das Rundumpaket welches wir den Schülern bieten wollen, zusätzlich bereichert werden. Der Nutzer kann im Stundenplan die Dauer, sowie die Anfangs- und Endzeit der jeweiligen Unterrichtseinheiten festlegen. Des Weiteren kann er dann die entsprechenden Einheiten im Stundenplan eintragen. Diese Angaben sind beliebig oft änderbar. Beim Klick auf eine der eingetragenen Unterrichtseinheiten soll sich dann automatisch das dazugehörige Heft des Users öffnen (wenn zuvor ein Heft dieser Unterrichtseinheit bzw. diesem Fach zugeteilt wurde).

Schüler sollten einfach Bilder hochladen, und den Text aus den Bildern automatisch in ihr Heft einfügen können. Dies sollte mit Optical Character Recognition (optische Zeichenerkennung) ermöglicht werden. Das OCR-System sollte ein Bild des Users entgegennehmen, dieses mit OCR-Algorithmen analysieren und schließlich den aus dem Bild erkannten Text in das Heft einfügen. Somit soll den Schülern ein vereinfachtes Einfügen von gedruckten Angaben oder Hausübungen, welche dann direkt bearbeitbar sind, ermöglicht werden.

Da Gruppenarbeiten und Referate im schulischen Bereich keine Seltenheit sind wurde auch an eine Optimierung in dieser Hinsicht gedacht. Durch ein eigenes ,,Parallel Working System" sollte das gemeinsame Arbeiten für Schüler deutlich vereinfacht werden. Jeder Schüler soll die Möglichkeit haben einige seiner Hefte mit anderen Schülern zu teilen. Diese geteilten Hefte können dann von allen dazu berechtigten Nutzern bearbeitet werden. Ein besonderes Augenmerk ist dabei auf die Konsistenz der bearbeiteten Hefte zu legen. Durch verschiedene Sperrmechanismen muss die Integrität der Daten in den geteilten Heften gesichert werden.

Um mit dem Produkt Geld zu verdienen sind, nach einer kostenlosen Probezeit von 90 Tagen, kostenpflichtige Benutzerkonten vorgesehen. Die Nutzer sollen also nach der Probezeit einen geringen monatlichen Betrag entrichten um das Produkt nutzen zu können. Zusätzlich ist der Vertrieb des Produkts an Schulen angedacht. Diese würden dann einen gewissen Betrag pro erworbener Lizenz zahlen, wobei diese Lizenz dann nicht zeitlich begrenzt ist und das Hosting auf den Servern der jeweiligen Schule stattfindet. Als zusätzliche Einnahmequelle könnte man noch, in Kooperation mit verschiedenen Werbenetzwerken, Werbung auf der Webseite schalten. Dabei wäre, aufgrund der vergleichsweise langen Verweildauer, die Seite der Heftansicht ein interessanter Platz für eine klassische Bannerwerbung oder Ähnliches.
