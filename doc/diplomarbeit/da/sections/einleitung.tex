%\section*{Einleitung}
\cfoot{Niklas Hohenwarter}

Am Anfang des Diplomprojektes Digital School Notes (\gls{DSN}) stand der Ideenfindungsprozess. Das Team wollte an einem sinnvollen Diplomprojekt arbeiten, also ein Projekt, welches nachher weitergeführt werden und verkauft werden könnte. Durch diese Art von Projekt versprach sich das Team eine höhere Arbeitsmoral. 

Die Idee kam während einer Unterrichtsstunde. Mal wieder schrieb niemand mit und kaum einer passte auf. Man hatte schon länger versucht eine Idee für ein Diplomprojekt zu finden und auf einmal war es doch so offensichtlich. Das Diplomprojekt sollte den Alltag in der Schule verbessern. Da kam die Idee für Digital School Notes.

Was, wenn mitschreiben auf einmal wieder interessanter ist? Wenn die Mitschrift am Laptop geführt wird und nicht mehr am Papier? Das würde so vieles einfacher machen. Ein digitales Schulheft. Sofort begannen wir die technische Machbarkeit zu prüfen. Die Software müsste eine Web Applikation sein um alle Betriebssysteme zu unterstützen. Sie müsste leicht zu bedienen und komfortabel sein. Es müsste möglich sein mehrere Schulen zu unterstützen, um eine große Userbase aufzubauen. Und natürlich müsste die Software attraktiv genug sein, um zahlende Kunden anzulocken und somit die Kosten der Infrastruktur decken zu können.

Nachdem die Idee fest stand wurde ein Lastenheft verfasst. Mit diesem Lastenheft hat das Team dann mehrere Lehrer kontaktiert, um weitere Ideen und Verbesserungsvorschläge einzuholen. Die Resonanz war gut, was das Team umso mehr motivierte. Das Projekt wurde genehmigt und die Implementierung begann.

Diese Diplomarbeit fasst die Arbeit am Projekt zusammen. Sie gibt Aufschluss über die Probleme und Lösungen, an welchen das Team gearbeitet hat.