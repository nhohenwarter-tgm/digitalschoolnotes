\section*{Kurzfassung}
\cfoot{Selina Brinnich}

Die heutige digitale Welt ist aus den Köpfen der meisten Menschen gar nicht mehr wegzudenken. Das Internet wird beinahe immer und überall verwendet - so auch immer mehr in Schulen. Vor allem in technischen Schulen werden Mitschriften aus dem Unterricht immer öfters digitalisiert. Das Problem dabei: Diese Mitschriften werden meist in unterschiedlichen Programmen verfasst, sie sind unorganisiert und verenden oft in irgendeinem Ordner ohne jemals wieder angesehen zu werden.\\
Das Projekt DigitalSchoolNotes setzt genau bei diesem Problem an. Unsere Web-Applikation soll das Führen einer digitalen Mitschrift einfacher und organisierter machen. Dabei wollen wir speziell auf die Bedürfnisse der Schüler eingehen. Der Zugriff auf die Applikation ist sowohl von Desktop Systemen und Laptops, als auch von Tablets über eine Webseite möglich. Ein eingeschränkter Zugriff ist zudem für Handys möglich, um beispielsweise Tafelbilder schnell und bequem in die Mitschrift einfügen zu können. Speziell für technisch orientierte Schulen gibt es auch die Möglichkeit, Teile von Programmcode richtig formatiert erstellen zu können. \\
Das alles dient dazu, Schülern eine ordentliche, organisierte, digitale Mitschrift zu erleichtern und das Lernen aus diesen einfacher und effizienter zu gestalten.