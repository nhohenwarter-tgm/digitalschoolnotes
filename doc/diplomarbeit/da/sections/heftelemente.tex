%\subsection*{Heftelemente}
\cfoot{}
Um Inhalt in ein Heft einzufügen gibt es die Ebene der Elemente. Diese verschiedenen Elemente bauen alle auf der selben Grundstruktur auf und sind ja nach Art des Elements weiter spezifiziert.
\insertpicture{images/elemente/elemente}{Heftelemente}{(selfmade)}{itm:elemente}{0.8}
Mittels Mausklick auf ein bestimmtes Icon auf der Toolbar wird das jeweilige Element in das Heft eingefügt.

Um das Element bearbeiten zu können, wird beim Überfahren des Elements ein Bleistift eingeblendet, welcher dem Benutzer erlaubt den bestehenden Inhalt des Elements zu editieren. Neben dem Bleistift erscheint ebenfalls ein Mistkübel, mit dem das Element entfernt werden kann.
Außerdem ist es möglich ein Element beliebig im Heft zu verschieben und dadurch beliebig im Heft zu platzieren. 

Diese Grundfunktionen sind in jedem Element enthalten. Jedes Element ist für den jeweiligen Anwendungsbereich genauer spezialisiert.